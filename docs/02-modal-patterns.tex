\hypertarget{sec:modal-fragment}{%
\section{The Modal Fragment}\label{sec:modal-fragment}}

The modal fragment of matching logic only allows quantifier- and fixedpoint-free
patterns and the empty theory.
This fragment may be regarded as a polyadic multiarity variant of modal logic.

\begin{definition}[The modal fragment]The \emph{modal fragment} of matching logic has:
\begin{align*}
\PP &= \{\text{ patterns built from \structure{} and \logic{} }\} \\
\TT &= \{\emptyset\}
\end{align*}\end{definition}

Our goal is to show the \emph{small model property} of the modal fragment of matching logic,
which states that if \(\psi\) is \emph{satisfiable},
then there exists a (finite) model \(M\) with size bounded by a computable
function \(f(\size{\psi})\) on the size of \(\psi\)
such that \(\evaluation{\phi}_M \neq \emptyset\)
(or equivalently, there is an element
\(a \in M\) such that \(a \vDash \psi\) in \(M\)).

The SMP implies decidability,
because one can exhaustively search for a model for \(\psi\)
up to the size bound \(f(\size{\psi})\).

\begin{remark}In the interest of space, all proofs for this section are in Appendix \ref{modal-appendix}\end{remark}

\begin{definition}[Closure]Given \(\psi\), let its \emph{closure} \(C(\psi)\) be the smallest set
that contains all its sub-patterns and their negations.\end{definition}

The size of \(C(\psi)\), written \(\size{C(\psi)}\),
is defined in the usual way and smaller than twice the size of \(\psi\).

\begin{definition}[\(\Gamma\)-indistinguishable]Given \(\Gamma\) and \(M\), we say that two elements \(a,b \in M\) are
\emph{\(\Gamma\)-indistinguishable}, written \(a \cong_\Gamma b\) or simply
\(a \cong b\) when \(\Gamma\) is understood,
if \(a \vDash \psi\) iff \(b \vDash \psi\) for all \(\psi \in \Gamma\).\end{definition}

\begin{lemma}\(\cong_\Gamma\) is an equivalence relation on \(M\).\end{lemma}

\begin{proof}By directly applying the definition.\end{proof}

\begin{definition}We use \([a]_\Gamma = \{b \in M \mid a \cong_\Gamma b\}\) to denote the
equivalence class of \(a \in M\).
We use \([A]_\Gamma = \{ [a]_\Gamma \mid a \in A \}\) to denote the set of
equivalence classes of all elements in \(A \subseteq M\).\end{definition}

In the following, we drop \(\Gamma\) when it is understood.

\begin{definition}\label{def:filtered}
Given $\psi$ and $M$, 
we consider $C(\psi)$-indistinguishability $\cong_{C(\psi)}$ or simply $\cong$.
Let $[a]$ be the equivalence class of 
$a \in M$.
Let the \emph{filtered model}
$[M]$ contain the following:
\begin{itemize}
\item the carrier set $[M]$;
\item the interpretation $\sigma_{[M]}\left([a_1],\dots,[a_n]\right) = 
[\sigma_M([a_1],\dots,[a_n])]$ for symbol $\sigma$ whose arity is $n$.
\end{itemize}
For any pattern $\phi$, we write $\phi_{[M]}$ to denote the 
interpretation of $\phi$ under any (irrelevant) valuation in $[M]$.  
\end{definition}

\begin{remark}
Note that $\sigma_{[M]}$ may not be functional, in the sense 
of \eqref{eq:functional-symbol}.
It can even evaluate to a set of more than one equivalent classes, 
even when $\sigma_M$ is functional.
\end{remark}

We emphasize that \([M]\) is indeed an matching logic model,
for which all it requires is a powerset interpretation of the symbols,
as done in Definition \ref{def:filtered}.
In fact, it is the nature of matching logic that makes this elegantly possible, because it
allows symbols to be interpreted in the powerset.
The above construction would not work for FOL, for example,
if \(\sigma\) were a function symbol.

We point out that \(\sigma_{[M]}([a_1],\dots,[a_n])\) is not necessarily
equal to \([\sigma_M(a_1,\dots,a_n)]\).
In general, \(\phi_{[M]}\)
is not necessarily equal to \([\phi_M]\) for arbitrary \(\phi\).
Later, in Lemma \ref{lemma:filter}, we show that \(\phi_{[M]} = [\phi_M]\)
holds for a selection of patterns, namely those which are in \(C(\psi)\).

\begin{lemma}\label{lem:filter_size}
$[M]$ has at most $\sqrt{2^{\size{C(\psi)}}}$ elements.
\end{lemma}

Note that \(\size{C(\psi)} \le 2{\size{\psi}}\) under a reasonable choice of the
size function.
Therefore,the space size of the filtered model \([M]\) is bounded by
\(\sqrt{2^{2\size{\psi}}} = 2^{\size{\psi}}\).

\begin{lemma}\label{lem:fl}
For any $\phi \in C(\psi)$, the following propositions are equivalent:
\begin{enumerate}
\item $\phi_{[M]} = [\phi_M]$;
\item $[a] \vDash \phi_{[{M}]}$ iff $a \vDash \phi_M$, for all $a \in M$.
\end{enumerate}
\end{lemma}

\begin{remark}
The condition $\phi \in C(\psi)$ is necessary in Lemma \ref{lem:fl}.
It is used in proving that (1) implies (2), for the ``only if'' direction.
\end{remark}

The following is the key lemma that links the semantics
between the original and the filtered models.

\begin{lemma}\label{lemma:filter}
For every $\phi \in C(\psi)$,  we have 
$\phi_{[M]} = [\phi_M]$.
\end{lemma}

\begin{theorem}\label{thm:ooosmp}
The modal fragment has the small model property. Formally, 
every satisfiable pattern $\psi$, without variables, $\exists$, or $\mu$,
has a model with at most ${2^{\size{\psi}}}$ 
elements.
\end{theorem}
\begin{proof}
Let $M$ and $a$ satisfy $a \vDash \psi$.
Applying Lemma \ref{lemma:filter} on $\psi$ and $a$,
we have that $[a] \vDash \phi$, in the filter model $[M]$,
which has at most ${2^{\size{\psi}}}$ 
elements, by Lemma \ref{lem:filter_size}.
\end{proof}

\begin{theorem}\label{thm:ooodec}
The modal fragment is decidable.
Formally, given a pattern $\psi$ without element variables, $\exists$, or 
$\mu$, determining whether $\psi$ is satisfiable is decidable.
\end{theorem}
\begin{proof}
By Theorem \ref{thm:ooosmp}, $\psi$ is satisfiable iff
there exist a model of size at most ${2^{\size{\psi}}}$.
Given any finite size $s$, there are only finitely many models of size $s$,
if we consider only the interpretations of the finitely many symbols that do 
occur in $\psi$.
Therefore, we can use a brute force procedure to enumerate
all possible models (with interpretations of symbols occur in $\psi$)
up to size ${2^{\size{\psi}}}$ and to check the satisfiability.
\end{proof}

We can extend Theorems \ref{thm:ooosmp} and \ref{thm:ooodec}
to patterns that also have \emph{set variables},
by replacing them by constant symbols.
Note that all set variables are free variables, because there is no binder
\(\mu\) in the fragment.

\begin{theorem}\label{thm:ooosmp_sv}
Every satisfiable pattern $\phi$, without element variables, $\exists$, or 
$\mu$,
has a model $M$ with at most ${2^{\size{\phi}}}$ elements,
i.e., there exists $\rho$ such that $\evaluation{\phi}_{M,\rho} \neq \emptyset$.
\end{theorem}
\begin{proof}
Let $X_1,\dots,X_k$ be all the set variables in $\psi$.
We define $k$ new constant symbols $c_1,\dots,c_k$
and let $\psi' \cong \psi[c_1/X_1]\cdots[c_k/X_k]$.
Note that $\size{C(\psi')} = \size{C(\psi)}$.
By Theorem \ref{thm:ooosmp}, $\psi'$ has a model $M$ with at most 
${2^{\size{\psi}}}$ elements, such that there exists $a \in M$
satisfying $a \vDash \psi'$.
We define a valuation $\rho$ such that $\rho(X_i) = (c_i)_{M'}$, for
every $1 \le i \le k$.
By matching logic semantics, $\evaluation{\psi'}_{\rho} = \evaluation{\psi}_{\rho}$,
so $a \in \evaluation{\psi}_{\rho}$.
Therefore, $\evaluation{\psi}_{\rho} \neq \emptyset$.
\end{proof}

\begin{theorem}\label{thm:ooodec_sv}
Given a pattern $\psi$, without element variables, $\exists$, or 
$\mu$,
determining whether $\psi$ is satisfiable is decidable.
\end{theorem}
\begin{proof}
The same as Theorem \ref{thm:ooodec}.
\end{proof}
