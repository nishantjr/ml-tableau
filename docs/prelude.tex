% Generic definitions
% ===================

% Fix bad defaults
% ----------------
 
% We prefer varphi over phi.
\renewcommand{\phi} {\varphi}
% By subset we normally mean non-strict subset.
\renewcommand{\subset}  {\subseteq}

% logic operators
% ---------------

\newcommand {\lAnd} {\bigwedge}
\newcommand {\lOr} {\bigvee}
\newcommand {\limplies}{\rightarrow}
\newcommand {\liff}{\leftrightarrow}

\newcommand {\free}[1] {\mathsf{free}(#1)}
\newcommand {\fnot}[1] {\mathsf{not}(#1)}

% set operators
% -------------

\newcommand {\Intersection}  {\bigcap}
\newcommand {\Union}         {\bigcup}
\newcommand {\intersection}  {\cap}
\newcommand {\union}         {\mathrel{\cup}}
\newcommand {\disjointunion} {\mathrel{\sqcup}}

\newcommand {\powerset} {\mathcal P}
\newcommand {\N} {\mathbb{N}}

% Model theory
% ------------
 
\newcommand {\proves}{\vdash}
\newcommand {\satisfies}{\models}

% Matching Logic
% --------------
 
\newcommand {\Var}      {\mathsf{Var}}
\newcommand {\EVar}     {\mathsf{EVar}}
\newcommand {\SVar}     {\mathsf{SVar}}
\newcommand{\evaluation}[1]{\left| {#1} \right|}

% K
% -

\newcommand {\K}  {$\mathbb{K}$}

% Environments
% ------------

\usepackage{amsthm}
\newtheorem{theorem}                 {Theorem} [section]
\newtheorem{proposition}  [theorem]  {Proposition}
\newtheorem{lemma}        [theorem]  {Lemma}
\theoremstyle{definition}
\newtheorem{definition}   [theorem]  {Definition}
\newtheorem{remark}       [theorem]  {Remark}
\newtheorem{example}      [theorem]  {Example}

% To do notes
% -----------

\usepackage{soul}
\newcommand {\todo}[1] {\hl{TODO: #1}}

% Paper Specific
% ==============
 
\usepackage{diagbox}

% Fragment Definitions
% --------------------

\usepackage{tikz}
\usetikzlibrary{shapes.misc, positioning}
\newcommand*\fragment[1]{
  \tikz[baseline=(char.base)]{\node[draw,rectangle,rounded corners=2pt,inner sep=2pt,fill=black,text=white] (char){\phantom{\rlap{fgq}}$\mathsf{#1}$};}
}
\newcommand{\structure}      { \fragment{structure} }
\newcommand{\logic}          { \fragment{logic} }
\newcommand{\quantification} { \fragment{quantification} }
\newcommand{\fixedpoint}     { \fragment{fixedpoint} }

% Signatures
\newcommand{\umeasure}{\mathsf{Measure}^\mu}
\newcommand{\vmeasure}{\mathsf{Measure}^\nu}


% Status Quo table
% ----------------

\usepackage{pifont}
\usepackage{newunicodechar}
\newunicodechar{✓}{\ding{51}}
\newunicodechar{✗}{\ding{55}}
\newunicodechar{⍻}{\ding{55}}
\newcommand{\cmark}{{\color{green}\ding{51}}}%
\newcommand{\xmark}{{\color{red}  \ding{55}}}%
\newcommand{\cxmark} {{\color{green}\ding{51}}\textsuperscript{\kern-0.47em\tiny\color{red}\ding{55}}}
\newcommand{\cqmark} {{\color{green}\ding{51}}\textsuperscript{\kern-0.47em\textbf{?}}}
\newcommand{\fin}{$_\mathrm{fin}$}
\renewcommand{\inf}{$_\mathrm{inf}$}

% Definition list, Tableau & Sequents
% -----------------------------------
 
\newcommand{\deflist}{\mathcal D}
\newcommand{\mkDeflist}[1]{\mathsf{deflist}(#1)}
\newcommand{\combineDefList}{\circ}

\newcommand{\matches}[2]{\mathsf{matches}(#1, #2)}
\newcommand{\sequent}[1]{\left\langle #1 \right\rangle}
\newcommand{\Sequent}{\mathsf{Sequent}}
\newcommand{\unsat}{\mathsf{unsat}}
\newcommand{\sat}{\mathsf{sat}}
\newcommand{\valid}{\mathsf{valid}}
\newcommand{\Basic}{\mathcal{B}}
\newcommand{\Universals}{\mathcal{U}}
\newcommand{\Elements}{\mathcal{E}}

\newcommand{\name}[1]{(\text{#1})\quad}
\newcommand{\dapp}   {$\overline{\mbox{app}}$}

\newcommand{\pruleun}[2]{\tabcolsep=2pt\begin{array}{c}%
 #1\\
 \noalign{\vskip-7pt}
 \hrulefill{{⓿}} \\
 \noalign{\vskip-2pt}
 #2\end{array}}

\newcommand{\prulebin}[3]{\frac{#1}{#2\quad#3}}

\newcommand{\satruleun}[2]
    {{\color{green}\pruleun{\color{black}#1}{\color{black}#2}}}
\newcommand{\satrulebin}[3]
    {{\color{green}\prulebin{\color{black}#1}{\color{black}#2}{\color{black}#3}}}

\newcommand{\unsatruleun}[2]
    {{\color{blue}\pruleun{\color{black}#1}{\color{black}#2}}}
\newcommand{\unsatrulebin}[3]
    {{\color{blue}\prulebin{\color{black}#1}{\color{black}#2}{\color{black}#3}}}

\newcommand{\inst}{\mathsf{inst}}

\newcommand{\vruleun}[2]{\frac{#2}{#1}}
\newcommand{\vrulebin}[3]{\frac{#2\quad#3}{#1}}



% Games
% -----

\newcommand{\Pos}{\mathrm{Pos}}

% Miscelanious
% ============

\newcommand{\FF}{\mathcal{F}}
\newcommand{\PP}{\mathcal{P}}
\newcommand{\TT}{\mathcal{T}}

\newcommand{\size}[1] {\left\lVert #1\right\rVert}
\newcommand{\width}[1]{\mathrm{width}(#1)}

\newcommand {\where}[1] {\quad\parbox[t]{0.5\textwidth}{where #1 }}
\newcommand {\since}[1] {\quad\parbox[t]{0.5\textwidth}{since #1 }}

\usepackage{multicol}

% End of LaTeX Prelude
% --------------------
