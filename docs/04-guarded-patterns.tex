\newcommand{\hideunlessappendix}[1]{}
\newcommand{\hideinappendix}[1]{#1}
\begin{figure*}
\footnotesize
$$\begin{array}{rlrlrrr}
\prule{conflict}                & \pruleun{\sequent{\{ \alpha \} \union \Gamma; \{ \fnot{\alpha} \} \union \Basic; \Universals}}
                                          { \unsat }
                                &
\prule{ok}                      & \pruleun{\sequent{\{ \alpha \} \union \Gamma; \{ \alpha \} \union \Basic; \Universals}}
                                          {\sequent{        \Gamma; \Basic; \Universals}}
\\
                                & \text{ when $\alpha$ is a basic assertion.}
                                &
                                & \text{ when $\alpha$ is a basic assertion.}
\\\\
\prule{conflict-el}             & \pruleun{\sequent{\matches{x}{y}, \Gamma; \Basic; \Universals}}
                                          { \unsat }
                                  \quad \text{when $x \neq y$.}
                                &
\prule{ok-el}                   & \pruleun{\sequent{\matches{x}{x}, \Gamma; \Basic; \Universals}}
                                          {\sequent{                \Gamma; \Basic; \Universals}} \\
\hideinappendix{ \\ \multicolumn{4}{c}{\text{\fbox{
Rules for \prule{and}, \prule{or}, \prule{def}, \prule{mu}, and \prule{nu}
identical to those in Figure~\ref{fig:qf-tableau} but lifted to the new form of sequents.
}}}\\\\}
\hideunlessappendix{
\prule{and}                     & \unsatruleun{\sequent{ \matches{z}{\phi} \land \matches{w}{\psi},   \Gamma; \Basic; \Universals}}
                                              {\sequent{ \matches{z}{\phi}, \matches{w}{\psi},  \Gamma; \Basic; \Universals}}
                                &
\prule{or}                      & \satrulebin{\sequent{ \matches{z}{\phi} \lor \matches{w}{\psi}, \Gamma; \Basic; \Universals}}
                                             {\sequent{ \matches{z}{\phi}, \Gamma; \Basic; \Universals}}
                                             {\sequent{ \matches{w}{\psi}, \Gamma; \Basic; \Universals}}
\\
\prule{def}                     & \pruleun{\sequent{ \matches{z}{\kappa X . \phi(X)}, \Gamma; \Basic; \Universals}}
                                          {\sequent{ \matches{z}{D}, \Gamma; \Basic; \Universals}}
\\
                                & \text{when $D := \kappa X. \phi(X) \in \deflist$ } \\
\prule{mu}                      & \pruleun{\sequent{ \matches{z}{D}, \Gamma; \Basic; \Universals}}
                                          {\sequent{ \matches{z}{\phi[D/X]}, \Gamma; \Basic; \Universals}}
                                &
\prule{nu}                      & \pruleun{\sequent{ \matches{z}{D}, \Gamma; \Basic; \Universals }}
                                          {\sequent{ \matches{z}{\phi[D/X]}, \Gamma; \Basic; \Universals}}
                                & \text{when $D := \nu X. \phi \in \deflist$ }
                                & \text{when $D := \mu X. \phi \in \deflist$ }
\\
}
\prule{forall}                  & \unsatruleun { \sequent{ \matches{z}{\forall \bar x \ldotp \phi}, \Gamma; \Basic; \Universals} }
                                               { \sequent{ \mathrm{inst} \union \Gamma
                                                         ; \Basic
                                                         ; \matches{z}{\forall \bar x \ldotp \phi}, \Universals
                                                         } }
                                &
\prule{\dapp}                   & \unsatruleun{\sequent{ \{\matches{z}{\bar\sigma(\bar \phi)}\} \union \Gamma; \Basic; \Universals}}
                                              { \sequent{ \mathrm{inst} \union \Gamma;
                                                          \Basic;
                                                          \{\matches{z}{\bar\sigma(\bar \phi)}\} \union \Universals;
                                              } }
\\
 & \text{where $\mathrm{inst} := \{ \matches{z}{ \phi[\bar y / \bar x]} \mid \bar y \subset \Elements \}$}
 &
 & \text{where $\matches{z}{\bar \sigma(\bar \phi)}$ is not a basic assertion.}
\\
 &
 &
 & \text{and $\mathrm{inst} := \left\{ \matches{z}{\fnot{\sigma(\bar y)}} \lor \lOr_i \matches{y_i}{\phi_i}
                                    \mid \bar y \subset \Elements \right\}$}
\\\\
\prule{choose-ex}               & \multicolumn{3}{l}{
                                  \unsatruleun {\sequent{ \Gamma; \Basic; \Universals }}
                                               {\{ \alpha \leadsto \sequent{ \Gamma;\Basic; \Universals } \mid \text{for each $\alpha \in \Gamma$}\}}
                                    \qquad\parbox{0.4\textwidth}{
                                    when all assertions in $\alpha$ are either existentials or applications. i.e. when no other of the above rules apply }}
\\\\
\prule{app}                     & \begin{array}{l}
                                  \satruleun { \matches{z}{\sigma(\bar \phi)} \leadsto \sequent{ \Gamma; \Basic; \Universals  } }
                                             { \left\{ \begin{array}{l}
                                                       \sequent{ \begin{array}{l}
                                                                     \matches{z}{\sigma(\bar k)}
                                                                     \\\land \lAnd_i \matches{k_i}{\phi_i}, \Gamma' \union \Universals'; \Basic' ; \emptyset
                                                                 \end{array}
                                                          }
                                                          \vspace{0.5em}\\
                                                          \text{for each $\bar k \subset \{z\} \union \free{\bar \phi} \union (K \setminus \Elements)$}
                                                        \end{array}
                                               \right\}
                                             } \\
                                  \text{where} \\
                                  \text{\qquad $\Elements' = \bar k \union \{ z \} \union  \free{\bar \phi}$} \\
                                  \text{\qquad $\Basic' = \Basic|_{\Elements'}$},
                                  \text{       $\Gamma' = \Gamma|_{\Elements'}$},
                                  \text{and    $\Universals' = \Universals|_{\Elements'}$} \\
                                  \end{array} &
\prule{exists}                  & \begin{array}{l}
                                  \satruleun { \matches{z}{\exists \bar x \ldotp \phi} \leadsto \sequent{ \Gamma; \Basic; \Universals } }
                                             { \{ \sequent{ \alpha, \Gamma' \union \Universals'; \Basic' ;  \emptyset } \}
                                             } \\
                                  \text{for each $\alpha \in \{ \matches{z}{\phi[\bar k/\bar x]} \mid \bar k \subset \{z\} \union \free{\bar \phi} \union (K \setminus \Elements) \}$} \\
                                  \text{where} \\
                                  \text{\qquad $\Elements'   = \free{\alpha}$} \\
                                  \text{\qquad $\Basic' = \Basic|_{\Elements'}$},
                                  \text{       $\Gamma' = \Gamma|_{\Elements'}$},
                                  \text{and    $\Universals' = \Universals|_{\Elements'}$}
                                  \end{array}
\\\\
%\cline{1-3}
%\intertext{This rule may only apply (as many times as needed) immediately after the (exists)/(app) rules or on the root node.
%$\mathsf{fresh}$ denotes the fresh variables introduced by the last application of those rules.}
\prule{resolve}                 & \multicolumn{3}{l}{
                                  \satrulebin{\sequent{ \Gamma; \Basic; \Universals}}
                                            {\sequent{ \Gamma; \matches{x_0}{\sigma(x_1,\ldots,x_n)}      \union \Basic; \Universals}}
                                            {\sequent{ \Gamma; \matches{x_0}{\lnot\sigma(x_1,\ldots,x_n)} \union \Basic; \Universals}} } \\
               & \multicolumn{3}{l}{\text{when neither $\matches{x}{{\sigma(x_1,\ldots,x_n)}}$ nor $\matches{x}{\fnot{\sigma(x_1,\ldots,x_n)}}$ are in $\Basic$}} \\
               & \multicolumn{3}{l}{\text{and  $\bar x \subset \Elements$ and $\bar x \intersection \mathsf{fresh} \neq \emptyset$.}} \\
               & \multicolumn{3}{l}{\text{May be applied only directly on the root node, or immediately after the application \prule{app}, \prule{exists} or \prule{resolve} rules.}} \\
\\
\end{array}$$
\caption{Tableau rules for the guarded fragment.}
\label{fig:guarded-tableau}
\end{figure*}



\hypertarget{sec:guarded-fragment}{%
\section{The Guarded Fragment}\label{sec:guarded-fragment}}

In this section, we present the guarded fragment of matching logic.
This fragment is inspired by the guarded fragment of fixedpoint logic\cite{guarded-fixedpoint-logic},
with extensions to allow for the differences between matching logic and fixedpoint logic.

Inspired by the robust decidablity properties of modal logic,
guarded logics were created as a means of ``taming'' a logic,
i.e.~of restricting a logic so that it becomes decidable.
This is done through syntactic restrictions on quantification.
In \cite{why-is-modal-logic-so-robustly-decidable}
we saw that the reason for this ``robust'' decidability was
that its models have the ``tree-model property'',
and that this property leads to automata based decision procedures.
With this insight, fragments that preserved decidability under such extensions were identified.
The \emph{guarded fragment} of first-order logic defined in \cite{modal-languages-and-bounded-fragments} allows
quantification over an arbitary number of variables so long as it is in the form
that, informally, relates each newly introduced variable to those previously
introduced.
This has since been generalized in two directions.
First, to allow more general guards.
In \emph{loosely guarded} first-order logic presented in \cite{loosely-guarded-fol}, guards are allowed to be conjunctions of atoms, rather than just single atoms.
\emph{Packed logic} extends this further, allowing even existentials to occur in guards.
In the \emph{clique guarded} fragment of first-order logic \cite{clique-guarded-logic}, quantification is semantically restricted to cliques within the Gaifman graph of models.
Second, to allow fixedpoints: guarded fixedpoint logic, loosely guarded fixedpoint logic \cite{guarded-fixedpoint-logic}, and clique-guarded fixedpoint logic \cite{clique-guarded-logic}.
extend the corresponding guarded logics to allow fixedpoints constructs.
An interesting property of guarded fixedpoint logics, is that despite being decidable, they admit ``infinity axioms'' --
axioms that are satisfiable only in infinite models.

The algorithm we present here is an extension to the one presented in \cite{guarded-fixedpoint-logic}
modified for matching logic with an important extension, to enable resolution,
that we found vital to a practical implementation.
We also try to empasize its relation with the tableau defined in Section \ref{sec:qf-fragment}.

For a nonempty tuple \(\bar x\),
We treat \(\exists \bar x \ldotp \phi\) and \(\forall \bar x \ldotp \phi\)
as shorthand for nested quantifier patterns.

\begin{definition}[Guarded pattern]A \emph{guarded pattern} is a closed (i.e.~without any free element or set variables)
positive-form pattern such that:

\begin{enumerate}
\def\labelenumi{\arabic{enumi}.}
\item
  Every existential sub-pattern is of the form \(\exists \bar x. \alpha \land \phi\)
  and every universal sub-pattern is of the form \(\forall \bar x. \alpha \limplies \phi\)
  where:

  \begin{enumerate}
  \def\labelenumii{\alph{enumii})}
  \tightlist
  \item
    \(\alpha\) is a conjunction where each conjunct is either an element variable,
    or an application pattern where each argument is an element variable,
  \item
    every variable \(x \in \bar x\) appears in some conjunct,
  \item
    for each pair of variables \(x \in \bar x\) and \(y \in \free{\phi}\)
    there is some application \(\sigma_i(\bar {z_i})\) in \(\alpha\) where \(x, y \in \bar {z_i}\).
    \label{gp:xxx}
  \end{enumerate}

  We call \(\alpha\) a guard.
\item
  If \(\sigma(\bar\phi)\) is an application, then
  each \(\phi_i\) is a conjunction of the form \(\xi \land \lAnd_{y \in \bar y} y\)
  where \(\bar y\) is a (possibly empty) set of element variables and \(\xi\) is a pattern,
  and every element variable in \(\free{\sigma(\phi_i)}\)
  is in \(\bar y\) for some \(\phi_i\).
\item
  \label{item:fixedpoint-no-elements}
  Each fixedpoint sub-pattern \(\mu X \ldotp \phi\) and \(\nu X \ldotp \phi\) has no free element variables.
\end{enumerate}

\end{definition}

\hypertarget{tableau-construction}{%
\subsection{Tableau Construction}\label{tableau-construction}}

While in the previous section, it was sufficient to use simple sets of patterns as sequents,
we now need something more complex.
Previously, each sequent in the quantifier-free tableau corresponds to a single
element in the model, in the more complex guarded patterns existentials
introduce additional elements we must keep track of.
We now build the nessesary constructs to represent those sequents.

\begin{definition}[Assertion]An \emph{assertion} is either:

\begin{enumerate}
\def\labelenumi{\arabic{enumi}.}
\tightlist
\item
  a pair of an element variable and a pattern, denoted \(\matches{x}{\phi}\),
\item
  a conjunction of assertions: \(\alpha_1 \land a_2\)
\item
  a disjunction of assertions: \(\alpha_1 \lor \alpha_2\)
\end{enumerate}

\end{definition}

Informally, assertions allow us to capture that a element in the model matches a pattern.

From here on, we treat \(\matches{x}{\phi_1\lor\phi_2}\) as equivalent to
\(\matches{x}{\phi_1} \lor \matches{x}{\phi_2}\).
and \(\matches{x}{\phi_1\land\phi_2}\) as equivalent to
\(\matches{x}{\phi_1} \land \matches{x}{\phi_2}\).

\begin{definition}[Basic assertions]\emph{Basic assertions} are assertions of the form:

\begin{enumerate}
\def\labelenumi{\arabic{enumi}.}
\tightlist
\item
  \(\matches{x_0}{\sigma(x_1, \ldots, x_n)}\),
\item
  \(\matches{x_0}{\bar\sigma(\lnot x_1, \ldots, \lnot x_n)}\).
\end{enumerate}

where each \(x_i\) is an element variable.\end{definition}

\emph{Basic} assertions, capture the relational interpretation of each symbol
and directly specify (portions) of the model.

\begin{definition}[Restriction]The \emph{free variables} of an assertion \(\matches{x}{\phi}\) are \(\{x\} \union \free\phi\).
For conjunctions and disjunctions of assertion, it is the union of the free variables
in each sub-pattern.
For a set of assertions \(A\) and a set of element variables \(E\),
the restriction of \(A\) to \(E\), denoted \(A|_{E}\),
is the subset of assertion in \(A\) whose free variables are a subset of \(E\).\end{definition}

The use of element variables in the \prule{\dapp} allow us to drop the
concept of \(\wit\).

\begin{definition}[Sequent]A \emph{sequent} is:

\begin{enumerate}
\def\labelenumi{\arabic{enumi}.}
\tightlist
\item
  a tuple, \(\sequent{\Gamma; \Basic; \Universals}\),
  where \(\Gamma\) is a set of assertions,
  \(\Basic\) is a set of basic assertions,
  \(\Universals\) is a set of assertions whose pattern is of the form \(\bar\sigma(...)\) or \(\forall \bar x\ldotp ...\).
\item
  \(\alpha \leadsto \sequent{\Gamma; \Basic; \Universals}\) where \(\alpha\) is an assertion
  and \(\Gamma, \Basic, \Universals\) are as above.
\item
  \(\unsat\)
\end{enumerate}

For a sequent \(v\) of the first two forms, we use \(\Gamma(v), \Basic(v)\) and \(\Universals(v)\)
to denote the corresponding component of the sequent.\end{definition}

Informally,
\(\Gamma\) represents the set of assertions whose combined satisfiability we are checking.
\(\Basic\) and \(\Universals\) are sets of consistent assertions that we are using to build our model.
Each free element variable in these assertions corresponds (roughly) to a distinct element in the
model we are building (if one exists).

\begin{definition}[Tableaux]\label{def:tableau}
Fix a definition list \(\deflist\) for \(\psi\).
A \emph{tableau} for \(\psi\) is a possibly infinite labeled tree \((T,L)\).
We denote its nodes as \(\Nodes(T)\) and the root node is \(\rt(T)\).
The labeling function \(L \colon \Nodes \to \mathsf{Sequents}\)
associates every node of \(T\) with a sequent, such that the following conditions
are satisfied:

\begin{enumerate}
\def\labelenumi{\arabic{enumi}.}
\tightlist
\item
  \(L(\rt(T)) = \{ \sequent{\matches{x}{\psi}} \}\) where \(x\) is a fresh element variable;
\item
  For every \(s \in \Nodes(T)\), if one of the tableau rules in \(\SSS\) in Figure \ref{fig:guarded-tableau} can be applied (with respect to the
  definition list \(\deflist^\psi\)), and the resulting sequents are
  \(\seq_1,\dots,\seq_k\), then
  \(s\) has exactly \(k\) child nodes \(s_1,\dots,s_k\), and
  \(L(s_1) = \seq_1\), \dots, \(L(s_k) = \seq_k\).
\end{enumerate}

\end{definition}

\begin{proposition}For any sequent built using these rules, we cannot have both \(\matches{x}{\phi}\) and \(\matches{x}{\fnot\phi}\) in \(\Basic\)\end{proposition}

\begin{proof}The root node starts with \(\Basic\) empty, and therefore this invariant is
maintained. Basic assertions are added to \(\Basic\) only through the
\prule{resolve} rule which maintains the invariant.\end{proof}

\begin{proposition}There is a tableau for any guarded pattern\end{proposition}

\begin{proof}For any assertion that is not basic, there is some rule that applies.
For a basic assertion if either the assertion itself or its negation in in \(\Basic\),
then the \prule{conflict} or \prule{ok} rule applies.
Otherwise, there was some application of the \prule{exists}, \prule{app} rule
after which all variables in the assertion are in \(\Elements\).
We may build a tableau where the \prule{resolve} rule
is applied for this basic assertion after this \prule{exists}/\prule{app} rule.\end{proof}

\hypertarget{parity-game}{%
\subsection{Parity Game}\label{parity-game}}

As before, using this tableau we now build a parity game.
In this game, player \(0\) may be thought of as trying to prove the satisfiablity of the pattern,
and player \(1\) as trying to prove it unsatisfiable.

Each position in the game is a pair \((a, v)\) where \(a\) is an assertion
and \(v\) is either a sequent or \(\sat\).
If \(v = \unsat\), then \(a\) is of the form \(\matches{x}{\bot}\).
If \(\Gamma = \emptyset\), then \(a\) is of the form \(\matches{x}{\top}\).
Otherwise, \(a \in \Gamma\).

There is an edge from a position \((a_0, v_0)\) to \((a_1, v_2)\) if:

\begin{enumerate}
\def\labelenumi{\alph{enumi})}
\item
  \(v_1 = \unsat\) is a child constructed through (conflict)
  and (conflict-el), \(a_0 = a_1\). (same as above)
\item
  \begin{enumerate}
  \def\labelenumii{\arabic{enumii}.}
  \tightlist
  \item
    \(v_1\) is constructed from \(v_0\)
    using the (and), (or), (def), (mu), (nu), (\dapp) or (forall) rules
    and \(a_0\) was modified by this rule,
    and \(a_1\) is one of the newly created assertions.
  \item
    \(v_0\)'s child is created using (ok) or (ok-el)
    and \(a_0 = a_1\) and \(v_1 = \sat\).
  \item
    \(v_1\) is constructed from \(v_0\)
    using (choose-ex)
    and \(a_0 = a_1 = \alpha\) is the matched existential.
  \item
    \(v_1\) is constructed from \(v_0\)
    using (app) or (exists)
    and \(a_0 = \alpha\) and \(a_1\) is the an instantiation from \(\inst\).
  \end{enumerate}
\item
  \(v_1\) is a child constructed through any rule besides (conflict)
  and (conflict-el),
  \(a_0 = a_1\) is in \(\Gamma(v_0) \union \Universals(v_0)\)
  and \(\Gamma(v_1) \union \Universals(v_1)\).
  and \(v_1 = v_0\)
\end{enumerate}

\newcommand{\green}[1]{{\color{green}#1}}
\newcommand{\blue}[1]{{\color{blue}#1}}

A position \(p = (a, v)\) is in \(\Pos_0\) by rules with a {\color{green}green} rule. That is, if:

\begin{enumerate}
\def\labelenumi{\arabic{enumi}.}
\tightlist
\item
  \(v\)'s children were built using (or), (app), or (exists) rules
  and \(a\) was the assertion matched on by that rule; or
\item
  \(v\)'s children were built using (resolve); or
\item
  \(v = \unsat\)
\end{enumerate}

A position \(p = (a, v)\) is in \(\Pos_1\) by rules with a {\color{blue}blue} rule. That is if:

\begin{enumerate}
\def\labelenumi{\arabic{enumi}.}
\tightlist
\item
  \(v\)'s children were built using (and), (\dapp), (forall), or (choose-ex) rules
  and \(a\) was the assertion matched on by that rule;
\item
  \(v = \sat\)
\end{enumerate}

All other positions do not offer a choice, and so are arbitrarily assigned to \(\Pos_1\).

The parity condition \(\Omega\) is defined as follows:

\begin{itemize}
\tightlist
\item
  \(\Omega((e \in D_X, v)) = 2 \times i\) if \(D_i\) is a \(\mu\)-marker that is \(i\)th in the dependency order.
\item
  \(\Omega((e \in D_X, v)) = 2 \times i + 1\) if \(D_i\) is a \(\nu\)-marker that is \(i\)th in the dependency order.
\item
  \(\Omega((e \in \exists \bar x\ldotp \phi, v)) = 2 \times N + 1\) where \(N\) is the number of fixedpoint markers in \(\deflist\).
\item
  \(\Omega(a, v) = 2 \times N + 2\) otherwise.
\end{itemize}

Similar to Theorems\textasciitilde{}\ref{thm:qf-decidable}
we may prove that pre-models correspond to models and that guarded patterns are decidable.

\begin{theorem}If a guarded pattern has a tableau with a pre-model, then it is satisfiable.\end{theorem}

\begin{theorem}\label{theorem:validity}
If a guarded pattern has a tableau with a refutation, then it is unsatisfiable
and its negation is valid.\end{theorem}

\hypertarget{working-modulo-theories}{%
\subsection{Working modulo theories}\label{working-modulo-theories}}

The tableau presented in Figure \ref{fig:guarded-tableau} may be easily extended to handle
satisfiability modulo theorems for finite theories with guarded axioms.

First, we extend assertions to allow quantifying over its variable---i.e.
we allow assertions of the form \(\forall x \ldotp \matches{x}{\phi}\).
Next, we add a new tableau rule:
\[
\text{\prule{axiom}}\qquad
\pruleun{\sequent{\forall x \ldotp \matches{x}{\phi},\Gamma;\Basic;\Universals}}
        {\sequent{\inst \union \Gamma;\Basic;\Universals}}
\]

for each axiom \(\tau\) in the theory and \(x \in \free{\Gamma\union\Basic\union\Universals}\).

\hypertarget{complexity}{%
\subsection{Complexity}\label{complexity}}
