% This is samplepaper.tex, a sample chapter demonstrating the
% LLNCS macro package for Springer Computer Science proceedings;
% Version 2.20 of 2017/10/04
%
\documentclass[sigplan,review,screen]{acmart}

% Generic definitions
% ===================

% Fix bad defaults
% ----------------
 
% We prefer varphi over phi.
\renewcommand{\phi} {\varphi}
% By subset we normally mean non-strict subset.
\renewcommand{\subset}  {\subseteq}

% logic operators
% ---------------

\newcommand {\lAnd} {\bigwedge}
\newcommand {\lOr} {\bigvee}
\newcommand {\limplies}{\rightarrow}
\newcommand {\liff}{\leftrightarrow}

\newcommand {\free}[1] {\mathsf{free}(#1)}
\newcommand {\fnot}[1] {\mathsf{not}(#1)}

% set operators
% -------------

\newcommand {\Intersection} {\bigcap}
\newcommand {\Union} {\bigcup}
\newcommand {\intersection} {\cap}
\newcommand {\union} {\mathrel{\cup}}

\newcommand {\powerset} {\mathcal P}
\newcommand {\N} {\mathbb{N}}

% Model theory
% ------------
 
\newcommand {\proves}{\vdash}
\newcommand {\satisfies}{\models}

% Matching Logic
% --------------
 
\newcommand {\Var}      {\mathsf{Var}}
\newcommand {\EVar}     {\mathsf{EVar}}
\newcommand {\SVar}     {\mathsf{SVar}}
\newcommand{\evaluation}[1]{\left| {#1} \right|}

% K
% -

\newcommand {\K}  {$\mathbb{K}$}

% Environments
% ------------

\usepackage{amsthm}
\newtheorem{theorem}                 {Theorem} [section]
\newtheorem{proposition}  [theorem]  {Proposition}
\newtheorem{lemma}        [theorem]  {Lemma}
\theoremstyle{definition}
\newtheorem{definition}   [theorem]  {Definition}
\newtheorem{remark}       [theorem]  {Remark}
\newtheorem{example}      [theorem]  {Example}

% To do notes
% -----------

\usepackage{soul}
\newcommand {\todo}[1] {\hl{TODO: #1}}

% Paper Specific
% ==============
 
\usepackage{diagbox}

% Fragment Definitions
% --------------------

\usepackage{tikz}
\usetikzlibrary{shapes.misc, positioning}
\newcommand*\fragment[1]{
  \tikz[baseline=(char.base)]{\node[draw,rectangle,rounded corners=2pt,inner sep=2pt,fill=black,text=white] (char){\phantom{\rlap{fgq}}$\mathsf{#1}$};}
}
\newcommand{\structure}      { \fragment{structure} }
\newcommand{\logic}          { \fragment{logic} }
\newcommand{\quantification} { \fragment{quantification} }
\newcommand{\fixedpoint}     { \fragment{fixedpoint} }

% Signatures
\newcommand{\umeasure}{\mathsf{Measure}_\mu}
\newcommand{\vmeasure}{\mathsf{Measure}_\nu}


% Status Quo table
% ----------------

\usepackage{pifont}
\usepackage{newunicodechar}
\newunicodechar{✓}{\ding{51}}
\newunicodechar{✗}{\ding{55}}
\newunicodechar{⍻}{\ding{55}}
\newcommand{\cmark}{{\color{green}\ding{51}}}%
\newcommand{\xmark}{{\color{red}  \ding{55}}}%
\newcommand{\cxmark} {{\color{green}\ding{51}}\textsuperscript{\kern-0.47em\tiny\color{red}\ding{55}}}
\newcommand{\cqmark} {{\color{green}\ding{51}}\textsuperscript{\kern-0.47em\textbf{?}}}
\newcommand{\fin}{$_\mathrm{fin}$}
\renewcommand{\inf}{$_\mathrm{inf}$}

% Definition list, Tableau & Sequents
% -----------------------------------
 
\newcommand{\deflist}{\mathcal D}
\newcommand{\mkDeflist}[1]{\mathsf{deflist}(#1)}
\newcommand{\combineDefList}{\circ}

\newcommand{\matches}[2]{\mathsf{matches}(#1, #2)}
\newcommand{\sequent}[1]{\left\langle #1 \right\rangle}
\newcommand{\Sequent}{\mathsf{Sequent}}
\newcommand{\unsat}{\mathsf{unsat}}
\newcommand{\valid}{\mathsf{valid}}
\newcommand{\Basic}{\mathcal{B}}
\newcommand{\Universals}{\mathcal{U}}
\newcommand{\Elements}{\mathcal{E}}

\newcommand{\name}[1]{(\text{#1})\quad}

\newcommand{\pruleun}[2]{\frac{#1}{#2}}
\newcommand{\prulebin}[3]{\frac{#1}{#2\quad#3}}

\newcommand{\satruleun}[2]
    {{\color{green}\pruleun{\color{black}#1}{\color{black}#2}}}
\newcommand{\satrulebin}[3]
    {{\color{green}\prulebin{\color{black}#1}{\color{black}#2}{\color{black}#3}}}

\newcommand{\unsatruleun}[2]
    {{\color{blue}\pruleun{\color{black}#1}{\color{black}#2}}}
\newcommand{\unsatrulebin}[3]
    {{\color{blue}\prulebin{\color{black}#1}{\color{black}#2}{\color{black}#3}}}

\newcommand{\inst}{\mathsf{inst}}

\newcommand{\vruleun}[2]{\frac{#2}{#1}}
\newcommand{\vrulebin}[3]{\frac{#2\quad#3}{#1}}



% Games
% -----

\newcommand{\Pos}{\mathrm{Pos}}

% Miscelanious
% ============

\newcommand{\FF}{\mathcal{F}}
\newcommand{\PP}{\mathcal{P}}
\newcommand{\TT}{\mathcal{T}}

\newcommand{\size}[1] {\left\lVert #1\right\rVert}
\newcommand{\width}[1]{\mathrm{width}(#1)}

\newcommand {\where}[1] {\quad\parbox[t]{0.5\textwidth}{where #1 }}
\newcommand {\since}[1] {\quad\parbox[t]{0.5\textwidth}{since #1 }}

% End of LaTeX Prelude
% --------------------

\usepackage{graphicx}
\makeatletter
\def\maxwidth{\ifdim\Gin@nat@width>\linewidth\linewidth\else\Gin@nat@width\fi}
\def\maxheight{\ifdim\Gin@nat@height>\textheight\textheight\else\Gin@nat@height\fi}
\makeatother
% Scale images if necessary, so that they will not overflow the page
% margins by default, and it is still possible to overwrite the defaults
% using explicit options in \includegraphics[width, height, ...]{}
\setkeys{Gin}{width=\maxwidth,height=\maxheight,keepaspectratio}

\graphicspath{{figs/}}

\newcommand{\tightlist}{}

\begin{document}

\title{Decidable Fragments of Matching Logic}
\begin{abstract}
    Matching logic has been put forward as a "lingua franca"
    for verifying and reasoning about programs and programming languages.
    We propose three increasingly powerful decidable fragments of matching logic.
    The largest of these fragments, called the guarded fragment,
    allows both fixedpoints and a restricted form of quantification.
    It is intended to extend the automated prover for uniform reasoning across
    logics in a previously developed framework,
    by providing a robust basis for unfolding fixedpoints and simplification.
\end{abstract}

\maketitle
\input{00-introduction.tex}
\hypertarget{sec:ml-prelims}{%
\section{Matching Logic Preliminaries}\label{sec:ml-prelims}}

Matching logic was first proposed in \cite{matchinglogiclmcs} as a unifying logic
for specifying and reasoning about programming languages.
An important feature of matching logic is that it makes no distinction between terms and formula.
This flexibility makes many important concepts easily definable in matching logic,
and allows for awkwardness free encodings of various abstractions and logics possible.
For example,
LTL formulae have identical syntax to their embedding in matching logic,
and unification may be characterized by conjuncting two pattern built from constructors.

Matching logic formulae are called \emph{patterns}
and have a ``pattern matching'' semantics,
in the sense that each pattern represents the set of elements that ``match'' it.
For example, \(\mathsf{cons}(42, x)\) matches lists whose first element is \(42\),
while \(\mathsf{prime} \land \mathsf{even}\) matches the natural \(2\),
assuming correct axiomatizations for \(\mathsf{cons}\), \(\mathsf{prime}\), and \(\mathsf{even}\).

\hypertarget{matching-logic-syntax}{%
\subsection{Matching Logic Syntax}\label{matching-logic-syntax}}

For a set \(\EVar\) of \emph{element variables}, denoted \(x, y, z, \ldots\),
and a set \(\SVar\) of \emph{set variables}, denoted \(X, Y, Z, \ldots\), we define the syntax of matching logic below.

\begin{definition}[Matching logic signatures]A matching logic \emph{signature}, \(\Sigma\) is a set of symbols with an associated arity.
Symbols with an arity of zero are called \emph{constants}.\end{definition}

\begin{definition}[Patterns]Given a signature \(\Sigma\), a countable set of element variables \(\EVar\) and of set variables \(\SVar\),
a matching logic \emph{pattern} is built recursively using the following grammar:
\begin{align*}
\phi:=  \underbrace{\sigma(\phi_1, \dots, \phi_n)}      _\text{\structure{}}
  \mid  \underbrace{\phi_1 \land \phi_2 \mid \lnot \phi}_\text{\logic{}}
  \mid  \underbrace{x \mid \exists x \ldotp \phi}       _\text{\quantification}
  \mid  \underbrace{X \mid     \mu X \ldotp \phi}       _\text{\fixedpoint{}}
\end{align*}
where \(x \in \EVar\), \(X \in \SVar\) and \(\sigma \in \Sigma\) has arity \(n\), and \(X\) occurs only positively in \(\mu X\ldotp \phi\). That is, \(X\) may only occur under an even number of negations in \(\phi\).\end{definition}

We assume the standard notions for free variables, \(\alpha\)-equivalence, and capture-free substitution \(\phi[\psi/x]\)
and allow the usual syntactic sugar:
\begin{align*}
                       \top &\equiv \exists x \ldotp x &
                       \bot &\equiv \lnot \top \\
         \phi_1 \lor \phi_2 &\equiv \lnot(\lnot\phi_1 \land \lnot\phi_2) &
    \phi_1 \limplies \phi_2 &\equiv \lnot \phi_1 \lor \phi_2 \\
      \forall x \ldotp \phi &\equiv \lnot \exists x \ldotp \phi &
      \nu X \ldotp \phi &\equiv \lnot \mu X \ldotp \lnot \phi[\lnot X/X]
\end{align*}
\(\sigma(\phi_1, \dots, \phi_n)\) are called applications.
Nullary applications are called constants, are denoted by using \(\sigma\) instead of \(\sigma()\).

\hypertarget{sec:semantics-of-matching-logic}{%
\subsection{Semantics of Matching Logic}\label{sec:semantics-of-matching-logic}}

Unlike in FOL, matching logic patterns are interpreted as a set of elements in a model rather than a single element.
Intuitively, the interpretation is the set of elements that match a pattern.
For example, the constant \(\mathsf{even}\) might have as interpretation the set of all even naturals,
while \(\mathsf{greaterThan}(3)\) may be interpreted as all integers greater than \(3\).
Function symbols may be considered a special case of this, where when applied to an argument the interpretation is a singleton set.
Logical constructs are thought of as set operations over matched elements
-- for example, \(\phi \land \psi\) is interpreted as the intersection of elements matched by \(\phi\) and \(\psi\),
while \(\lnot \phi\) matches all elements \emph{not} matched by \(\phi\).
An existential \(\exists x \ldotp \phi(x)\) is interpreted as the union of all patterns matching \(\phi(x)\) for all valuations of \(x\).
\(\mu X \ldotp \phi(X)\) matches the \emph{least} set \(X\) such that \(X\) and \(\phi(X)\) match the same elements.
An important point to note here is that element variables have as evaluation exactly a single element,
whereas set variables may be interpreted as any subset of the carrier set.

\begin{definition}[\(\Sigma\)-models]Given a signature \(\Sigma\), a \(\Sigma\)-model is a tuple \((\mathbb M, \{ \sigma_M \}_{\sigma \in \Sigma} )\)
where \(\mathbb M\) is a set of elements called the carrier set,
and \(\sigma_M : M^n \to \powerset(M)\) is the interpretation of the symbol \(\sigma\) with arity \(n\) into the powerset of \(M\).\end{definition}

We use \(M\) to denote both the model \(M\), and it's carrier set, \(\mathbb M\).
We also tacitly use \(\sigma_M\) to denote the pointwise extension, \(\sigma_M : \powerset(M)^n \to \powerset(M)\),
defined as \(\sigma_M(A_1,\dots,A_n) \mapsto \Union_{a_i\in A_i} \sigma_M(a_1,\dots,a_n)\)
for all sets \(A_i \subseteq M\).

\begin{definition}[Semantics of matching logic]\label{def:semantics}Let \(\rho : \EVar{} \union \SVar{} \to \powerset(M)\) be a function such that \(\rho(x)\) is a singleton set when \(x \in \EVar\),
called an evaluation.
Then, the evaluation of a pattern \(\phi\), written \(\evaluation{\phi}_{M,\rho}\) is defined inductively by:
\begin{align*}
\evaluation{\sigma(\phi_1, \ldots, \phi_n)}_\rho &= \sigma_M(\evaluation{\phi_1}, \ldots, \evaluation{\phi_n}) \text{ for $\sigma$ of arity $n$} \\
\evaluation{\phi_1 \land \phi_2}_\rho            &= \evaluation{\phi_1}_\rho \intersection \evaluation{\phi_2}_\rho \\
\evaluation{\lnot \phi}_\rho                     &= M\setminus \evaluation{\phi}_\rho \\ 
\evaluation{x}_\rho                              &= \rho(x) \text{ for } x \in \EVar \\
\evaluation{\exists x \ldotp \phi}_\rho          &= \bigcup_{a \in M}  \evaluation{\phi}_{\rho[a/x]}\\
\evaluation{X}_\rho                              &= \rho(X) \text{ for } X \in \SVar  \\
\evaluation{\mu X \ldotp \phi}_\rho              &= \mathsf{LFP}(\FF)
\end{align*}
\begin{align*}
\text{where }&&
    \FF(A) &= \evaluation{\phi}_{\rho[A/X]} \text{ for } A \subseteq M, \\
\text{and} &&
    \mathsf{LFP}(f) &\mapsto \Intersection\left\{A \in \powerset{M} \mid f(A) \subset A \right\}
\end{align*}
takes a monotonic function to its least fixedpoint \cite{matching-mu-logic}.\end{definition}

As seen, \(\sigma\) is interpreted as a relation.
Its interpretation \(\sigma_M\) is not a function in the standard FOL sense.
We say that \(\sigma_M\) is \emph{functional}, if:
\begin{equation}\tag{functional-symbol}
\label{eq:functional-symbol}
\size{\sigma_M(a_1,\dots,a_n)} = 1  \quad \text{for all $a_1 \in  M_{s_1}, \dots, a_n \in M_{s_n}$}
\end{equation}

\hypertarget{satisfiability-and-validity}{%
\subsection{Satisfiability and Validity}\label{satisfiability-and-validity}}

In this subsection, formally define satisfiability and validity in matching logic\footnote{
Note that our definitions differ from \cite{matchinglogiclmcs} where only validity in a model is defined (but referred to as satisfiability).
We avoid using the $\models$ notation to avoid confusion between the two.
}.
Because of the powerset interpretation of patterns, the notions of satisfiability and validity differ
subtly from those in FOL.
The interpretations of FOL sentences are two-valued---they must be true or false.
This means that the notions of satisfiability and validity in a model coincide.
However, in Matching logic patterns evaluate to a subset of the carrier set.
We say a pattern is satisfiable in a model when its evaluation is non-empty,
and that it is valid when its evaluation is the entire carrier set.
In particular, even for closed patterns both a pattern and its negation may be satisfiable.
For example, the model \(\mathbb N\) with the usual interpretations,
satisfies both \(\mathsf{even}\) and \(\lnot \mathsf{even}\) (i.e.~the set of odd naturals) but neither are valid.

\begin{definition}[Satisfiability in a model]
We say a $\Sigma$-model $M$ \emph{satisfies} a $\Sigma$-pattern
iff there is some evaluation $\rho$ and an element $m$
such that $m \in \evaluation{\phi}_{M,\rho}$.
A $\Sigma$-pattern $\phi$ is \emph{satisfiable} iff there is a model $M$ that satisfies $\phi$.
\end{definition}

\begin{definition}[Validity in a model]
We say a $\Sigma$-pattern is \emph{valid} in a $\Sigma$-model $M$
iff for all evaluations $\rho$, $\evaluation{\phi}_{M,\rho} = M$.
\end{definition}

Analogously to FOL, we may define theories in matching logic.
Essentially, a theory is a set of patterns, called axioms, that are valid in a model.
A pattern is satisfiable modulo a theory if it is satisfiable in some model where all axioms are valid.

\begin{definition}[Satisfiability modulo theories]
Let $\Gamma$ be a set of $\Sigma$-patterns called \emph{axioms}.
We say $\phi$ is satisfiable modulo theory $\Gamma$ if there is a model $M$
such that each $\gamma$ in $\Gamma$ is valid and $M$ satisfies $\phi$.
\end{definition}

\begin{definition}[Validity modulo theories]Let \(\Gamma\) be a set of \(\Sigma\)-patterns called \emph{axioms}.
We say \(\phi\) is satisfiable modulo theory \(\Gamma\)
if for all models \(M\)
such that each \(\gamma\) in \(\Gamma\) is valid we have \(\phi\) is valid in \(M\).\end{definition}

\begin{remark}[A note about variants of matching logic]In its original formulation, matching logic had a many-sorted flavor where each symbol and pattern had a fixed sort.
While it is convenient to define models that are also many-sorted,
in \cite{applicative-matching-logic} the authors point out that
the many-sorted setting actually becomes an obstacle when it comes to
more complex sort structures.
Therefore, they proposed a much simpler, unsorted variant of matching logic called applicative matching logic (AML),
where the many-sorted infrastructure is dropped and sorts are instead defined axiomatically.
This also treated multi-arity applications, as syntactic sugar for nested applications.
In this work, to maximize the expressivity of the fragment defined here
while still avoiding the complexity of multiple sorts,
we use a version of matching logic that sits between the two,
allowing multi-arity applications, but without sorts.
When we need to be explicit about this distinction, we will refer to this as \emph{polyadic matching logic}.
For the rest of this document unless explicitly mentioned,
we will use pattern, model, etc, to refer to those concepts in polyadic matching logic
although the same terms may be used in other variants of matching logic.\end{remark}

\hypertarget{fragments-and-meta-properties}{%
\subsection{Fragments and Meta-Properties}\label{fragments-and-meta-properties}}

In general, matching logic's power and expressivity entails
that the logic as a whole does not have some desirable properties.
For example, because it subsumes first-order logic, the satisfiability problem must be undecidable.
Further, because we can precisely pin down the standard model of the natural
numbers using the fixedpoint operator, by G\"odel's incompleteness theorem, it
must also be incomplete.

When studying such properties in the context of matching logic, we must thus restrict ourselves to subsets of matching logic.
In this section, we shall formally define what we mean by a ``fragment'' of matching logic,
and define some properties we care about.

\begin{definition}[Fragments of matching logic]A \emph{fragment of matching logic} is a pair \((\PP, \TT)\)
where \(\PP\) is a set of patterns and \(\TT\) is a set of theories.
We say a pattern \(P\) is in a fragment if \(P \in \PP\),
and a theory \(\Gamma\) is in a fragment if \(\Gamma \in \TT\)\end{definition}

Fragments may be defined with any number of criteria,
including the restrictions on
the use of quantifiers and fixedpoints,
number and arity of symbols,
the number of axioms,
quantifier alternation and so on.

We will now define the properties of fragments of matching logic that we will study in this document.

\begin{definition}[Decidable fragment]A fragment of matching logic, \((\PP, \TT)\), is \emph{decidable}
if there is an algorithm for deterimining the satisfiability of any pattern \(P \in \PP\) in any theory \(\Gamma \in \TT\) in the fragment.\end{definition}

Notice that if \(\PP\) is closed under negation, then the validity problem for a decidable fragment is also decidable.

For proving the decidability of some fragments in this paper, we rely on a more specific property called the small-model property.
This property says that every \(\Gamma\)-satisfiable pattern in a fragment has a model bound by a computable function on the size of the pattern.
Formally:

\begin{definition}[Small-model property]A fragment of matching logic, \((\PP, \TT)\), has the small-model property iff for every pattern \(P \in \PP\) in every theory \(\Gamma \in \TT\)
if \(P\) is \(\Gamma\)-satisfiable then, there is some model \(M \satisfies \phi\) whose size is bound by a computable function \(f\) on the size of \(\phi\).
That is, \(\size{M} \le f(\size{\phi})\).\end{definition}

The small-model property implies that a fragment is decidable since one may simply
enumerate all models of size up to \(f(\size{\phi})\) and evaluations
and check satisfiablility in each of them.
The small-model property is a stonger version of another interesting property, called the finite-model property:

\begin{definition}[Finite-model property]A fragment of matching logic, \((\PP, \TT)\), has the finite-model property iff for every pattern \(P \in \PP\) in every theory \(\Gamma \in \TT\)
if \(P\) is \(\Gamma\)-satisfiable then, there is some model \(M \satisfies \phi\) with finite size.\end{definition}

The finite-model property and decidablity are independent in the sense that a
fragment may have the finite model property and yet be undecidable, or be
decidable despite being infinite.

In the next sections,
we will define some fragments and prove some properties about them.

We summarize the meta properties of these fragments in Table \ref{table:status-quo}.

\begin{table*}
\small
\begin{tabular} {|r||l|l|l||l|l|l||l|l|l|}
\hline
              & \multicolumn{3}{c||}{Empty theories}                                                                                    &\multicolumn{3}{c||}{Finite theories}                                                                                                  & \multicolumn{3}{c|}{Recursively enumerable theories}  \\
\hline
\diagbox[height=2em,width=9em]{Property}{Fragment}
              &  Modal                                  & Quant. free                       & Guarded                                   &  Modal                                      & Quant. free                                 & Guarded                                   &  Modal                                  & Quant. free & Guarded                                   \\
\hline\rule{0pt}{3ex}
Small-model   & \cmark[Sec.\ref{sec:modal-fragment}]    & \cmark[Sec.\ref{sec:qf-fragment}] & \xmark                                    & \qmark                                      & \qmark                                      & \xmark                                    & \xmark                                  & \xmark      & \xmark                                    \\
Finite-model  & \cmark                                  & \cmark                            & \xmark[Ex.\ref{ex:naturals-are-guarded}]  & \qmark                                      & \qmark                                      & \xmark                                    & \xmark[Ex. \ref{ex:modal-inf-infinite}] & \xmark      & \xmark                                    \\
Decidability  & \cmark                                  & \cmark                            & \cmark                                    & \cmark[Sec.\ref{sec:guarded-fragment}]      & $\dagger$[Sec.\ref{sec:guarded-fragment}]   & \cmark[Sec.\ref{sec:guarded-fragment}]    & \xmark\cite{urquhart1981}               & \xmark      & \xmark                                    \\
\hline
\end{tabular}
\caption{
  \emph{The status quo:} Fragments of matching logic and their meta-prorties. \newline
  $\dagger$ This result has only been proved when there are no free set variables in axioms. \newline
}
\label{table:status-quo}
\end{table*}

\input{02-modal-patterns.tex}
\hypertarget{sec:qf-fragment}{%
\section{The Quantifier-Free Fragment}\label{sec:qf-fragment}}

The quantifier free fragment is less restrictive, allowing fixedpoints in patterns as well:

\begin{definition}[Quantifier-Free Patterns]The \emph{quantifier-free patterns} of matching logic has:

\begin{align*}
\PP &= \left\{\begin{array}{l}\text{ patterns built from \structure{}, } \\
            \text{\logic{} and \fixedpoint{}}
              \end{array}\right\}\\
\TT &= \{\emptyset\}
\end{align*}\end{definition}

This fragment also exhibits the small-model property as proved in {[}@sec:decidable-qf-fragment{]}.

In this section, we prove that quantifier-free fragment is decidable and has
the small model property.
We do this by reducing the satisfiability problem to a solving a ``parity game'' (a decidable problem).
Given a pattern, the parity game is built from a ``tableau''.
The tableau is a possibly infinite tree constructed using a set of syntax driven rules.
Although the tree itself is infinite,
its labels range over a finite set of labels that repeat along infinite paths in a ``regular'' way
and so has a finite representation.

In both this section and the next,
we define our procedure over ``positive-form'' patterns
-- patterns where negations are pushed down as far as they can go using
De Morgan's and related equivalences.

\begin{definition}[Positive-Form Patterns]Positive form patterns are defined using the syntax:

\begin{alignat*}{5}
\phi := \quad&       \sigma(\phi_1, \ldots, \phi_n)
   &\quad\mid\quad&  \bar \sigma(\phi_1, \ldots, \phi_n) \\
    \quad\mid\quad&  \phi_1 \land \phi_2
   &\quad\mid\quad&  \phi_1 \lor  \phi_2 \\
    \quad\mid\quad&  x
   &\quad\mid\quad&  \lnot x
   &\quad\mid\quad&  \exists x \ldotp \phi
   &\quad\mid\quad&  \forall x \ldotp \phi \\
    \quad\mid\quad&  X &
                  &    &
    \quad\mid\quad&  \mu X \ldotp \phi
   &\quad\mid\quad&  \nu X \ldotp \phi
\end{alignat*}
where \(\bar \sigma(\phi_1, \ldots, \phi_n) \equiv \lnot\sigma(\lnot \phi_1, \ldots, \lnot\phi_n)\).\end{definition}

When \(\sigma\) is a nullary symbol we use \(\sigma\) and \(\bar \sigma\) as shorthand for \(\sigma()\) and \(\bar \sigma()\).

We allow negation of element variables, but not set variables.
This ensures that set variables may only occur positively in their binding fixedpoint.
While positive form patterns allow existentials and universals,
in the fragment under consideration in this section we do not allow them.

By definition, we have:

\begin{lemma}
Every pattern is equivalent to some positive-form pattern.
\end{lemma}

From now on, we only consider positive-form patterns and simply call them
patterns.

\hypertarget{defintion-lists-and-dependency-orders}{%
\subsection{Defintion lists and dependency orders}\label{defintion-lists-and-dependency-orders}}

\begin{definition}[Definition Lists]For a quantifier-free pattern \(\phi\), a \emph{definition list}
(denoted \(\deflist^\phi\) or just \(\deflist\) when \(\phi\) is understood)
is a mapping from each occurring set variable to the pattern by which it is bound.
Since we assume well-named patterns, each set variable is bound by a unique
fixedpoint pattern and such a mapping is well-defined.
We use \(\deflist^\phi(X)\) to denote the fixedpoint sub-pattern corresponding to the set variable \(X\).\end{definition}

\begin{definition}[Fixedpoint Markers]For a variable \(X \in \dom(\deflist^\phi)\),
\(\deflist^\phi_X\) (or just \(\deflist_X\) when \(\phi\) is understood)
is a \emph{fixedpoint marker}.
We call a marker a \(\mu\)-marker if \(\deflist^\phi(X)\) is a \(\mu\) pattern
and a \(\nu\)-marker otherwise.
We extend the syntax of patterns to allow fixedpoint markers. These markers may
be used whereever set variables may be used -- in particular, they may not
appear negated in positive-form patterns. We define the evaluation of fixedpoint
makers as the evaluation of their corresponding fixedpoint pattern:
\[\evaluation{\deflist^\phi_X}^\deflist_{M,\rho} = \evaluation{\deflist^\phi(X)}_{M,\rho}\]\end{definition}

Since the evaluation of a pattern now also depends on its dependency list,
we make the dependency list used explicit by adding it as a superscript as in \(\evaluation{\phi}^\deflist_{M,\rho}\).

\begin{figure*}
\footnotesize
$$\def\arraystretch{2.5}\begin{array}{rlrl}
\prule{conflict}
& \pruleun{\sigma,\bar\sigma,\Gamma}
  { \unsat }
\\
\prule{and} 
& \pruleun{\phi_1 \wedge \phi_2,\Gamma}
                {\phi_1,\phi_2,\Gamma}
&\prule{or}
& \prulebin{\phi_1 \vee \phi_2, \Gamma}
                 {\phi_1,\Gamma}
                 {\phi_2,\Gamma}
\\
\prule{unfold}
&
\multicolumn{3}{l} {
 \pruleun{U, \Gamma}
                {\phi[U/X], \Gamma}
\where{$(\deflist_U = \kappa X \ldotp \phi)$ \\
         and $\kappa \in \{\mu,\nu\}$}
} \\
\prule{mu}
& \pruleun{\mu X \ldotp \phi, \Gamma}{U,\Gamma}
\where{ $(\deflist_U = \mu X \ldotp \phi)$}

&\prule{nu}
&  \pruleun{\nu X \ldotp \phi, \Gamma}{U,\Gamma}
\where{$(\deflist_U = \nu X \ldotp \phi)$}
\\
\appa
& \multicolumn{3}{l} {
\pruleun {\Gamma}
                 {\left\{ \sigma(\phi_1,\dots,\phi_n) \leadsto \Gamma_{\bar\sigma}
                  \mid \sigma(\phi_1,\dots,\phi_n) \in \Gamma
                  \right\} }
                 {
                 \where{
                    $\Gamma$ contains only
                    $\sigma$-patterns, and $\bar\sigma$-patterns.
                    (In other words, only if all other rules cannot be applied.)
                 }
                 }
}
\\
\appb
&
\pruleun{\sigma(\phi_1,\dots,\phi_n) \leadsto \Gamma }
        {\left\{ \sigma(\phi_1,\dots,\phi_n) \leadsto \wit \leadsto \Gamma \mid \wit \in 
         \Wit(\Gamma,\sigma) \right\}}
\\
\appc
& \pruleun{\sigma(\phi_1,\dots,\phi_n) \leadsto \wit \leadsto \Gamma}
{\left\{ \phi_i, \Gamma^\wit_i \mid i \in \{1,\dots,n\} \right\}}
\end{array}$$
\caption{Tableau rules for the quantifier-free fragment.}
\label{fig:qf-tableau}
\end{figure*}

\begin{definition}[Depends On]For two fixedpoint markers \(\deflist^\phi_X\) and \(\deflist^\phi_Y\),
we say \(\deflist^\phi_X\) \emph{depends on} \(\deflist^\phi_Y\)
if \(\deflist^\phi(Y)\) is a sub-pattern of \(\deflist^\phi(X)\).\end{definition}

The transitive closure of this relation is a pre-order -- i.e.~it is reflexive and transitive.
It is also antisymmetric -- for a pair of distinct markers \(\deflist^\phi_X\) and \(\deflist^\phi_Y\)
we may have either \(\deflist^\phi_X\) transitively depends on \(\deflist^\phi_Y\)
or \(\deflist^\phi_Y\) transitively depends on \(\deflist^\phi_X\) but not both.
So, it is also a partial order.

\begin{definition}[Dependency Orders]For a pattern \(\phi\), a \emph{dependency order},
is an (arbitary) extension of the above partial order to a total (linear) order.\end{definition}

Note that the partial order may extend to several total orders.
For defining our parity game, it does not matter which one we choose so long as it is a total order.
So, through abuse of notation, we use \(\dependson\) to denote some arbitary dependency order.

\begin{example}For the pattern, \(\nu X \ldotp (s(X) \land \mu Y \ldotp z \lor s(Y))\),
we have \[\deflist^\phi = 
\begin{cases}
X &\mapsto \nu X \ldotp (s(X) \land \mu Y \ldotp z \lor s(Y)) \\
Y &\mapsto \mu Y \ldotp z \lor s(Y)
\end{cases}\]

and a dependency order: \(X \dependson Y\).\end{example}

\begin{example}For the pattern, \(\nu X \ldotp s(X) \land \bar p \land \mu Y \ldotp s(Y) \land p\),
we have \[\deflist^\phi = 
\begin{cases}
X &\mapsto \nu X \ldotp s(X) \land \bar p \\
Y &\mapsto \mu Y \ldotp s(Y) \land p
\end{cases}\]

and dependency order: \(X \dependson Y\).
However, this isn't the only dependency order -- we also have \(Y \dependson X\).\end{example}

\hypertarget{tableau-construction}{%
\subsection{Tableau Construction}\label{tableau-construction}}

\begin{definition}
Let $\Gamma$ be a set of patterns.
We define $\Gamma_\sigma$ (resp. $\Gamma_{\bar\sigma}$) to be the set of 
$\sigma$-patterns 
(resp. $\bar\sigma$-patterns) in $\Gamma$.
Formally,
\begin{align*}
\Gamma_\sigma &= \{ 
\sigma(\phi_1,\dots,\phi_n) \mid \sigma(\phi_1,\dots,\phi_n) \in 
\Gamma \} \\
\Gamma_{\bar\sigma} &= \{ 
\bar\sigma(\phi_1,\dots,\phi_n) \mid \bar\sigma(\phi_1,\dots,\phi_n) \in 
\Gamma \}
\end{align*}
\end{definition}

\begin{definition}
Given $\Gamma$ and any non-constant symbol $\sigma$ of arity $n \ge 1$,
we define
a \emph{witness function} $\wit \colon \Gamma_{\bar\sigma} \to \{1,\dots,n\}$
as a function that maps every pattern of the form 
$\bar\sigma(\psi_1,\dots,\psi_n) 
\in \Gamma$ with a number between $1$ and $n$, called the \emph{witness}.
Let $\Wit(\Gamma,\sigma) = [\Gamma_{\bar\sigma} \to \{1,\dots,n\}]$ denote the set 
of all witness functions with respect to $\Gamma$ and $\sigma$.
Let $\Gamma^\wit_i = \{\psi_i \mid \bar\sigma(\psi_1,\dots,\psi_n) \in \Gamma 
\text{ and } \wit(\bar\sigma(\psi_1,\dots,\psi_n)) = i \}$.
\end{definition}

\begin{remark}
When $\Gamma_{\bar\sigma} = \emptyset$, we assume there is a unique witness 
function denoted $\wit_\emptyset$, whose domain is empty. 
This is mainly for technical convenience. 
\end{remark}

We use \(\size{A}\) to denote the cardinality of any set \(A\).

\begin{remark}
Suppose $\sigma$ has arity $n$ and $\size{\Gamma_{\bar\sigma}} = m$  is finite.
Then $\size{\Wit(\Gamma,\sigma)} = n^m$.
In particular, if $\sigma$ is a unary symbol, i.e., $n = 1$, then 
$\size{\Wit(\Gamma,\sigma)} = 1$. 
\end{remark}

\begin{definition}A \emph{tableau sequent} is one of the following:

\begin{enumerate}
\def\labelenumi{\arabic{enumi}.}
\tightlist
\item
  a finite nonempty pattern set \(\Gamma\);
\item
  \(\Gamma \leadsto \sigma(\phi_1,\dots,\phi_n)\), where
  \(\sigma(\phi_1,\dots,\phi_n) \in \Gamma\);
\item
  \(\Gamma \leadsto \wit \leadsto \sigma(\phi_1,\dots,\phi_n)\), where
  \(\sigma(\phi_1,\dots,\phi_n) \in \Gamma\)
  and \(\wit \in \Wit(\Gamma,\sigma)\).
\item
  \(\unsat\)
\end{enumerate}

\end{definition}

\begin{definition}[Tableaux]\label{def:tableau}
Fix a definition list \(\deflist\) for \(\psi\).
A \emph{tableau} for \(\psi\) is a possibly infinite labeled tree
\((T,L)\).
We denote its nodes as \(\Nodes(T)\) and the root node is \(\rt(T)\).
The labeling function \(L \colon \Nodes \to \mathsf{Sequents}\)
associates every node of \(T\) with a sequent, such that the following conditions
are satisfied:

\begin{enumerate}
\def\labelenumi{\arabic{enumi}.}
\tightlist
\item
  \(L(\rt(T)) = \{ \psi \}\);
\item
  For every \(s \in \Nodes(T)\), if one of the tableau rules in \(\SSS\) in {[}@fig:qf-tableau{]} can be applied (with respect to the
  definition list \(\deflist^\psi\)), and the resulting sequents are\\
  \(\seq_1,\dots,\seq_k\), then
  \(s\) has exactly \(k\) child nodes \(s_1,\dots,s_k\), and
  \(L(s_1) = \seq_1\), \dots, \(L(s_k) = \seq_k\).
\end{enumerate}

\end{definition}

In (3), we categorize the nodes by the corresponding tableau rules that are
applied.
For example, if the two child nodes of \(s\) is obtained by applying \prule{or},
then we call \(n\) an \prule{or} node.

\begin{remark}
For any tableau $(T,L)$ and an \appa{} node $s \in \Nodes(T)$,
its child nodes (if there are any) must be \appb{} nodes, 
whose child nodes must be \appc{} nodes.
\appb{} and \appc{} nodes must have at least one child nodes,
i.e., they are not leaf nodes. 
\end{remark}

\begin{remark}
For any tableau $(T,L)$, the leaves of $T$
are either labeled with inconsistent sequents, or they are
\appa{} nodes whose labels contain no $\sigma$-patterns for any 
$\sigma$.
For any non-leaf node, unless it is labeled with \prule{or} or \prule{app$_i$} 
for $i \in \{1,2,3\}$,
it has exactly one child node. 
\end{remark}

\hypertarget{parity-games}{%
\subsection{Parity games}\label{parity-games}}

Now that we know how to construct a tableau for a quantifier-free pattern,
we can derive a parity game from it.

\begin{definition}[Parity Games]A \emph{parity game} is a tuple \((\Pos_0, \Pos_1, E, \Omega)\)
where \(\Pos = \Pos_0 \disjointunion \Pos_1\) is a possibly infinite set of positions;
Each \(\Pos_i\) is called the \emph{winning set} for player \(i\).
\(E : \Pos \times \Pos\) is a transition relation;
and \(\Omega : \Pos \to \N\) defines the \emph{parity winning condition}
mapping each position to a natural number below some finite bound.\end{definition}

The game is played between two players, player \(0\) and player \(1\).
When the game is in a position \(p \in \Pos_i\) then it is player \(i\)'s turn -- i.e.~player \(i\) may choose
the next node to transition to form the current node, along the transition relation \(E\).

\begin{definition}[Plays]Each game results in a (possible infinite) sequence of positions, called plays: \(p_0, p_1, p_2,...\).
A play is finite if a player is unable to make a move. In that case that player loses.
For an infinite play, we look at the sequence of parities of the vertices in the play
-- i.e.~\(\Omega(p_0), \Omega(p_1), ...\).
Player \(0\) wins iff the least parity that occurs infinitely often is even, otherwise player \(1\) wins.\end{definition}

\begin{definition}For a \((\Pos_0, \Pos_1, E, \Omega)\),
a \emph{strategy} from a position \(q\) for a player \(i\) is a subtree \((P \subset \Pos, S \subset E)\) such that:

\begin{enumerate}
\def\labelenumi{\arabic{enumi}.}
\tightlist
\item
  \(q \in P\)
\item
  If a node \(p \in P \intersection \Pos_i\)
  then there is \emph{exactly} one edge outward edge \((p, p')\) in \(S\) and \(p' \in P\).
\item
  If a node \(p \not\in \Pos_i\) and \(p \in P\) then all outward edges from \(p\) in \(E\) are in \(S_i\) and \(p \in W_i\).
\end{enumerate}

A strategy is \emph{winning} for a player \(i\) if Player \(i\) wins on every trace.\end{definition}

\todo{ (cite: Infinite games on finitely coloured graphs with applicatiosn to automata on infinite trees)}
\todo{ (cite: Tree automata, mu-calculus and determinacy) }

\todo{define memoryless strategy}

In the case of the particular parity game we define for quantifier-free patterns,
one may think of player \(0\) as trying to search for a model and player \(1\)
as trying to show that there cannot be a model.

\begin{definition}Let \(\mathcal T = (T, L)\) be a tableau for a quantifier-free pattern \(\psi\).
Below, we define a parity game \(\game = (\PosT_0, \PosT_1, \ET, \OmegaT)\).
The positions are the set of pairs from \(\Pattern \times (\{\sat\} \union \Sequent)\).
The positions and edge relation are inductively defined as below:

\begin{itemize}
\tightlist
\item
  there is a position \(p = (\psi, \rt(T))\) corresponding to the root sequent.
\item
  If \(p = (\phi, s)\) is a position, and \(s'\) is a child of \(s\) in \(T\)

  \begin{itemize}
  \tightlist
  \item
    if \(s\) is not \appc{} node, then

    \begin{itemize}
    \tightlist
    \item
      there is a positon \(p' = (\phi, s')\) with \((p, p') \in E\)
      if the rule does not reduce \(\phi\);
    \item
      for each \(\phi'\) in the set of reductions of \(\phi\) in \(s'\)
      there is a positon \(p' = (\phi', s')\) with \((p, p') \in E\).
    \end{itemize}
  \item
    if \(s\) is an \appc{} node with \(s = \sigma(\phi_1,\ldots,\phi_n) \leadsto \wit \leadsto \Gamma\), then

    \begin{itemize}
    \tightlist
    \item
      if \(\phi = \sigma(\phi_1,\ldots,\phi_n)\)
      then there is a position \(p' = (\phi_i, s')\) with \((p, p') \in E\)
    \item
      if \(\phi = \bar\sigma(\chi_1,\ldots,\chi_n)\)
      and \(\wit(\phi) = i\)
      then there is a position \(p' = (\phi_i, s')\) with \((p, p') \in E\)
    \end{itemize}
  \item
    if a position has no children as defined by the above rules
    then there is a position \(p' = (\phi, \sat)\) with \((p, p') \in E\) .
    (this is the case when a \(\bar\sigma-\)pattern is dropped by all children of a \appa{} node
    or when \(\appc\) is applied to a nullary \(\sigma\) pattern)
  \end{itemize}
\end{itemize}

These positions are partitioned into \(\Pos_0\) and \(\Pos_1\) as follows:

\begin{itemize}
\tightlist
\item
  A position \(p = (\phi, s)\) is in \(\Pos_0\) if either:

  \begin{itemize}
  \tightlist
  \item
    \(s\) is an (or)- or \appb{}-node \(\phi\) and \(\phi\) was transformed by the rule
  \item
    \(s = \unsat{}\)
  \end{itemize}
\item
  A position \(p = (\phi, s)\) is in \(\Pos_1\) if either:

  \begin{itemize}
  \tightlist
  \item
    \(s\) is an (and)-, \appa{}-, or \appc{}-node \(\phi\) and \(\phi\) was transformed by the rule
  \item
    \(s = \sat{}\)
  \end{itemize}
\item
  All other positions offer no choice---they have exactly one outward edge.
  We arbitarily assign this to \(\Pos_1\).
\end{itemize}

We define the parity condition \(\Omega\):

\begin{itemize}
\tightlist
\item
  \(\Omega((\nu X\ldotp \phi, s)) = 2i\) if \(\deflist_X\) is \(i\)th in the dependency order
\item
  \(\Omega((\mu X\ldotp \phi, s)) = 2i + 1\) if \(\deflist_X\) is \(i\)th in the dependency order
\item
  \(\Omega((\phi, s)) = 2N + 2\) if where \(N\) is the size of \(\deflist\)
\end{itemize}

\end{definition}

By this definition, all leaf positions are either \(\sat{}\) or \(\unsat{}\).

\begin{definition}[Pre-models \& Refutations]We call a winning strategy for player \(0\) a pre-model,
and a winning strategy for player \(1\) a refutation.\end{definition}

Any play consistent with a pre-model must terminate with \(\sat{}\) if finite.
An infinite play must have an even number as lowest parity---i.e.~the priority corresponding to some \(\nu\) fixedpoint marker.
Any play consistent with a winning strategies for player \(1\) terminate with \(\unsat{}\) if finite.
An infinite play must have an odd number as lowest parity---i.e.~the priority corresponding to some \(\mu\) fixedpoint marker.

We show that for any quantifier-free sentence \(\psi\), it is satisfied in a
model (i.e., its interpretation is nonempty) iff a pre-model exists for \(\psi\).
In the interest of space we only give an informal overview of the proof here,
and refer the reader to the {[}Appendix @sec:qf-proofs{]} for the proofs in their
entirety.

Our first objective is to prove that if there is a satisfying model, then
every tableau for \(\psi\) contains a pre-model as a sub-tree.

\begin{proposition}
\label{prop:mpm}
If a quantifier-free sentence $\psi$ is satisfied in $M$ on $a$, then
any tableau for $\psi$ contains a pre-model for $\psi$ as a sub-tree. 
\end{proposition}

We do this by constructing a strategy for player \(0\) while maintaining the
invariant that the patterns labeling each position reachable using the strategy
are satisfied by some element in the model.
We show that this strategy is winning for player \(0\), i.e.~a pre-model.
This is done by showing the every move taken must either reduces the size of the patterns in the sequent
or a measure called the \(\umeasure\), corresponding roughly to the minimum number of unfoldings of \(\mu\) patterns needed to satisfy a pattern in the model,
unless the move is unfolding a \(\nu\)-pattern.

Next, we show that if we have a pre-model for some quantifier-free pattern,
then that pattern must be satisfiable.

\begin{proposition}\label{prop:pmm}
If there exists a pre-model for a positive-from quantifier-free sentence $\psi$
then $\psi$ is satisfiable in the corresponding canonical model. 
\end{proposition}

In this case we construct a model, called the cannonical model, from the pre-model.
We then assume that the model does not satisfy the pattern, by way of contradiction,
and show that there must be a \(\mu\)-play in the strategy if this model does
not satisfy the pattern.

\begin{theorem}\label{thm:pm}
For any positive-form quantifier-free sentence $\psi$, there exists a pre-model for $\psi$ 
iff $\psi$ is satisfiable.
\end{theorem}
\begin{proof}
By Propositions \ref{prop:mpm} and \ref{prop:pmm}. 
\end{proof}

\begin{theorem}\label{thm:qf-decidable}
For any quantifier-free pattern $\psi$, determining whether $\psi$ is 
satisfiable is decidable.
\end{theorem}
\begin{proof}
By Theorem \ref{thm:pm}, we can look for a pre-model for $\psi$.
We will construct a tree automaton $\Aut$ on infinite trees that accepts 
exactly the pre-models for $\psi$.
Then by the Emerson-Jutla theorem, it is decidable to determine whether the 
language accepted by $\Aut$ is empty. 

$\Aut$ is constructed in two steps. Firstly, we define an \emph{outer automaton}
$\Aut_o$ that accepts the quasi-models, which are essentially the \emph{regular 
trees} generated by the set of tableau rules $\SSSmod$.
Secondly, we define an \emph{inner automaton} $\Aut_i$ that is a B\"uchi 
automaton on infinite words that accepts the $\mu$-traces.
Then, we combine the two automatons using the Safra deterministic construction
and obtain a tree automaton that accepts precisely the pre-models for $\psi$.
\end{proof}

\todo{talk about refutations}

\newcommand{\hideunlessappendix}[1]{}
\newcommand{\hideinappendix}[1]{#1}
\begin{figure*}
\footnotesize
$$\begin{array}{rlrlrrr}
\prule{conflict}                & \pruleun{\sequent{\{ \alpha \} \union \Gamma; \{ \fnot{\alpha} \} \union \Basic; \Universals}}
                                          { \unsat }
                                &
\prule{ok}                      & \pruleun{\sequent{\{ \alpha \} \union \Gamma; \{ \alpha \} \union \Basic; \Universals}}
                                          {\sequent{        \Gamma; \Basic; \Universals}}
\\
                                & \text{ when $\alpha$ is a basic assertion.}
                                &
                                & \text{ when $\alpha$ is a basic assertion.}
\\\\
\prule{conflict-el}             & \pruleun{\sequent{\matches{x}{y}, \Gamma; \Basic; \Universals}}
                                          { \unsat }
                                  \quad \text{when $x \neq y$.}
                                &
\prule{ok-el}                   & \pruleun{\sequent{\matches{x}{x}, \Gamma; \Basic; \Universals}}
                                          {\sequent{                \Gamma; \Basic; \Universals}} \\
\hideinappendix{ \\ \multicolumn{4}{c}{\text{\fbox{
Rules for \prule{and}, \prule{or}, \prule{def}, \prule{mu}, and \prule{nu}
identical to those in Figure~\ref{fig:qf-tableau} but lifted to the new form of sequents.
}}}\\\\}
\hideunlessappendix{
\prule{and}                     & \unsatruleun{\sequent{ \matches{z}{\phi} \land \matches{w}{\psi},   \Gamma; \Basic; \Universals}}
                                              {\sequent{ \matches{z}{\phi}, \matches{w}{\psi},  \Gamma; \Basic; \Universals}}
                                &
\prule{or}                      & \satrulebin{\sequent{ \matches{z}{\phi} \lor \matches{w}{\psi}, \Gamma; \Basic; \Universals}}
                                             {\sequent{ \matches{z}{\phi}, \Gamma; \Basic; \Universals}}
                                             {\sequent{ \matches{w}{\psi}, \Gamma; \Basic; \Universals}}
\\
\prule{def}                     & \pruleun{\sequent{ \matches{z}{\kappa X . \phi(X)}, \Gamma; \Basic; \Universals}}
                                          {\sequent{ \matches{z}{D}, \Gamma; \Basic; \Universals}}
\\
                                & \text{when $D := \kappa X. \phi(X) \in \deflist$ } \\
\prule{mu}                      & \pruleun{\sequent{ \matches{z}{D}, \Gamma; \Basic; \Universals}}
                                          {\sequent{ \matches{z}{\phi[D/X]}, \Gamma; \Basic; \Universals}}
                                &
\prule{nu}                      & \pruleun{\sequent{ \matches{z}{D}, \Gamma; \Basic; \Universals }}
                                          {\sequent{ \matches{z}{\phi[D/X]}, \Gamma; \Basic; \Universals}}
                                & \text{when $D := \nu X. \phi \in \deflist$ }
                                & \text{when $D := \mu X. \phi \in \deflist$ }
\\
}
\prule{forall}                  & \unsatruleun { \sequent{ \matches{z}{\forall \bar x \ldotp \phi}, \Gamma; \Basic; \Universals} }
                                               { \sequent{ \mathrm{inst} \union \Gamma
                                                         ; \Basic
                                                         ; \matches{z}{\forall \bar x \ldotp \phi}, \Universals
                                                         } }
                                &
\prule{\dapp}                   & \unsatruleun{\sequent{ \{\matches{z}{\bar\sigma(\bar \phi)}\} \union \Gamma; \Basic; \Universals}}
                                              { \sequent{ \mathrm{inst} \union \Gamma;
                                                          \Basic;
                                                          \{\matches{z}{\bar\sigma(\bar \phi)}\} \union \Universals;
                                              } }
\\
 & \text{where $\mathrm{inst} := \{ \matches{z}{ \phi[\bar y / \bar x]} \mid \bar y \subset \Elements \}$}
 &
 & \text{where $\matches{z}{\bar \sigma(\bar \phi)}$ is not a basic assertion.}
\\
 &
 &
 & \text{and $\mathrm{inst} := \left\{ \matches{z}{\fnot{\sigma(\bar y)}} \lor \lOr_i \matches{y_i}{\phi_i}
                                    \mid \bar y \subset \Elements \right\}$}
\\\\
\prule{choose-ex}               & \multicolumn{3}{l}{
                                  \unsatruleun {\sequent{ \Gamma; \Basic; \Universals }}
                                               {\{ \alpha \leadsto \sequent{ \Gamma;\Basic; \Universals } \mid \text{for each $\alpha \in \Gamma$}\}}
                                    \qquad\parbox{0.4\textwidth}{
                                    when all assertions in $\alpha$ are either existentials or applications. i.e. when no other of the above rules apply }}
\\\\
\prule{app}                     & \begin{array}{l}
                                  \satruleun { \matches{z}{\sigma(\bar \phi)} \leadsto \sequent{ \Gamma; \Basic; \Universals  } }
                                             { \left\{ \begin{array}{l}
                                                       \sequent{ \begin{array}{l}
                                                                     \matches{z}{\sigma(\bar k)}
                                                                     \\\land \lAnd_i \matches{k_i}{\phi_i}, \Gamma' \union \Universals'; \Basic' ; \emptyset
                                                                 \end{array}
                                                          }
                                                          \vspace{0.5em}\\
                                                          \text{for each $\bar k \subset \{z\} \union \free{\bar \phi} \union (K \setminus \Elements)$}
                                                        \end{array}
                                               \right\}
                                             } \\
                                  \text{where} \\
                                  \text{\qquad $\Elements' = \bar k \union \{ z \} \union  \free{\bar \phi}$} \\
                                  \text{\qquad $\Basic' = \Basic|_{\Elements'}$},
                                  \text{       $\Gamma' = \Gamma|_{\Elements'}$},
                                  \text{and    $\Universals' = \Universals|_{\Elements'}$} \\
                                  \end{array} &
\prule{exists}                  & \begin{array}{l}
                                  \satruleun { \matches{z}{\exists \bar x \ldotp \phi} \leadsto \sequent{ \Gamma; \Basic; \Universals } }
                                             { \{ \sequent{ \alpha, \Gamma' \union \Universals'; \Basic' ;  \emptyset } \}
                                             } \\
                                  \text{for each $\alpha \in \{ \matches{z}{\phi[\bar k/\bar x]} \mid \bar k \subset \{z\} \union \free{\bar \phi} \union (K \setminus \Elements) \}$} \\
                                  \text{where} \\
                                  \text{\qquad $\Elements'   = \free{\alpha}$} \\
                                  \text{\qquad $\Basic' = \Basic|_{\Elements'}$},
                                  \text{       $\Gamma' = \Gamma|_{\Elements'}$},
                                  \text{and    $\Universals' = \Universals|_{\Elements'}$}
                                  \end{array}
\\\\
%\cline{1-3}
%\intertext{This rule may only apply (as many times as needed) immediately after the (exists)/(app) rules or on the root node.
%$\mathsf{fresh}$ denotes the fresh variables introduced by the last application of those rules.}
\prule{resolve}                 & \multicolumn{3}{l}{
                                  \satrulebin{\sequent{ \Gamma; \Basic; \Universals}}
                                            {\sequent{ \Gamma; \matches{x_0}{\sigma(x_1,\ldots,x_n)}      \union \Basic; \Universals}}
                                            {\sequent{ \Gamma; \matches{x_0}{\lnot\sigma(x_1,\ldots,x_n)} \union \Basic; \Universals}} } \\
               & \multicolumn{3}{l}{\text{when neither $\matches{x}{{\sigma(x_1,\ldots,x_n)}}$ nor $\matches{x}{\fnot{\sigma(x_1,\ldots,x_n)}}$ are in $\Basic$}} \\
               & \multicolumn{3}{l}{\text{and  $\bar x \subset \Elements$ and $\bar x \intersection \mathsf{fresh} \neq \emptyset$.}} \\
               & \multicolumn{3}{l}{\text{May be applied only directly on the root node, or immediately after the application \prule{app}, \prule{exists} or \prule{resolve} rules.}} \\
\\
\end{array}$$
\caption{Tableau rules for the guarded fragment.}
\label{fig:guarded-tableau}
\end{figure*}



\hypertarget{sec:guarded-fragment}{%
\section{The Guarded Fragment}\label{sec:guarded-fragment}}

In this section, we present the guarded fragment of matching logic.
This fragment is inspired by the guarded fragment of fixedpoint logic\cite{guarded-fixedpoint-logic},
with extensions to allow for the differences between matching logic and fixedpoint logic.

Inspired by the robust decidablity properties of modal logic,
guarded logics were created as a means of ``taming'' a logic,
i.e.~of restricting a logic so that it becomes decidable.
This is done through syntactic restrictions on quantification.
In \cite{why-is-modal-logic-so-robustly-decidable}
we saw that the reason for this ``robust'' decidability was
that its models have the ``tree-model property'',
and that this property leads to automata based decision procedures.
With this insight, fragments that preserved decidability under such extensions were identified.
The \emph{guarded fragment} of first-order logic defined in \cite{modal-languages-and-bounded-fragments} allows
quantification over an arbitary number of variables so long as it is in the form
that, informally, relates each newly introduced variable to those previously
introduced.
This has since been generalized in two directions.
First, to allow more general guards.
In \emph{loosely guarded} first-order logic presented in \cite{loosely-guarded-fol}, guards are allowed to be conjunctions of atoms, rather than just single atoms.
\emph{Packed logic} extends this further, allowing even existentials to occur in guards.
In the \emph{clique guarded} fragment of first-order logic \cite{clique-guarded-logic}, quantification is semantically restricted to cliques within the Gaifman graph of models.
Second, to allow fixedpoints: guarded fixedpoint logic, loosely guarded fixedpoint logic \cite{guarded-fixedpoint-logic}, and clique-guarded fixedpoint logic \cite{clique-guarded-logic}.
extend the corresponding guarded logics to allow fixedpoints constructs.
An interesting property of guarded fixedpoint logics, is that despite being decidable, they admit ``infinity axioms'' --
axioms that are satisfiable only in infinite models.

The algorithm we present here is an extension to the one presented in \cite{guarded-fixedpoint-logic}
modified for matching logic with an important extension, to enable resolution,
that we found vital to a practical implementation.
We also try to empasize its relation with the tableau defined in Section \ref{sec:qf-fragment}.

For a nonempty tuple \(\bar x\),
We treat \(\exists \bar x \ldotp \phi\) and \(\forall \bar x \ldotp \phi\)
as shorthand for nested quantifier patterns.

\begin{definition}[Guarded pattern]A \emph{guarded pattern} is a closed (i.e.~without any free element or set variables)
positive-form pattern such that:

\begin{enumerate}
\def\labelenumi{\arabic{enumi}.}
\item
  Every existential sub-pattern is of the form \(\exists \bar x. \alpha \land \phi\)
  and every universal sub-pattern is of the form \(\forall \bar x. \alpha \limplies \phi\)
  where:

  \begin{enumerate}
  \def\labelenumii{\alph{enumii})}
  \tightlist
  \item
    \(\alpha\) is a conjunction where each conjunct is either an element variable,
    or an application pattern where each argument is an element variable,
  \item
    every variable \(x \in \bar x\) appears in some conjunct,
  \item
    for each pair of variables \(x \in \bar x\) and \(y \in \free{\phi}\)
    there is some application \(\sigma_i(\bar {z_i})\) in \(\alpha\) where \(x, y \in \bar {z_i}\).
    \label{gp:xxx}
  \end{enumerate}

  We call \(\alpha\) a guard.
\item
  If \(\sigma(\bar\phi)\) is an application, then
  each \(\phi_i\) is a conjunction of the form \(\xi \land \lAnd_{y \in \bar y} y\)
  where \(\bar y\) is a (possibly empty) set of element variables and \(\xi\) is a pattern,
  and every element variable in \(\free{\sigma(\phi_i)}\)
  is in \(\bar y\) for some \(\phi_i\).
\item
  \label{item:fixedpoint-no-elements}
  Each fixedpoint sub-pattern \(\mu X \ldotp \phi\) and \(\nu X \ldotp \phi\) has no free element variables.
\end{enumerate}

\end{definition}

\hypertarget{tableau-construction}{%
\subsection{Tableau Construction}\label{tableau-construction}}

While in the previous section, it was sufficient to use simple sets of patterns as sequents,
we now need something more complex.
Previously, each sequent in the quantifier-free tableau corresponds to a single
element in the model, in the more complex guarded patterns existentials
introduce additional elements we must keep track of.
We now build the nessesary constructs to represent those sequents.

\begin{definition}[Assertion]An \emph{assertion} is either:

\begin{enumerate}
\def\labelenumi{\arabic{enumi}.}
\tightlist
\item
  a pair of an element variable and a pattern, denoted \(\matches{x}{\phi}\),
\item
  a conjunction of assertions: \(\alpha_1 \land a_2\)
\item
  a disjunction of assertions: \(\alpha_1 \lor \alpha_2\)
\end{enumerate}

\end{definition}

Informally, assertions allow us to capture that a element in the model matches a pattern.

From here on, we treat \(\matches{x}{\phi_1\lor\phi_2}\) as equivalent to
\(\matches{x}{\phi_1} \lor \matches{x}{\phi_2}\).
and \(\matches{x}{\phi_1\land\phi_2}\) as equivalent to
\(\matches{x}{\phi_1} \land \matches{x}{\phi_2}\).

\begin{definition}[Basic assertions]\emph{Basic assertions} are assertions of the form:

\begin{enumerate}
\def\labelenumi{\arabic{enumi}.}
\tightlist
\item
  \(\matches{x_0}{\sigma(x_1, \ldots, x_n)}\),
\item
  \(\matches{x_0}{\bar\sigma(\lnot x_1, \ldots, \lnot x_n)}\).
\end{enumerate}

where each \(x_i\) is an element variable.\end{definition}

\emph{Basic} assertions, capture the relational interpretation of each symbol
and directly specify (portions) of the model.

\begin{definition}[Restriction]The \emph{free variables} of an assertion \(\matches{x}{\phi}\) are \(\{x\} \union \free\phi\).
For conjunctions and disjunctions of assertion, it is the union of the free variables
in each sub-pattern.
For a set of assertions \(A\) and a set of element variables \(E\),
the restriction of \(A\) to \(E\), denoted \(A|_{E}\),
is the subset of assertion in \(A\) whose free variables are a subset of \(E\).\end{definition}

The use of element variables in the \prule{\dapp} allow us to drop the
concept of \(\wit\).

\begin{definition}[Sequent]A \emph{sequent} is:

\begin{enumerate}
\def\labelenumi{\arabic{enumi}.}
\tightlist
\item
  a tuple, \(\sequent{\Gamma; \Basic; \Universals}\),
  where \(\Gamma\) is a set of assertions,
  \(\Basic\) is a set of basic assertions,
  \(\Universals\) is a set of assertions whose pattern is of the form \(\bar\sigma(...)\) or \(\forall \bar x\ldotp ...\).
\item
  \(\alpha \leadsto \sequent{\Gamma; \Basic; \Universals}\) where \(\alpha\) is an assertion
  and \(\Gamma, \Basic, \Universals\) are as above.
\item
  \(\unsat\)
\end{enumerate}

For a sequent \(v\) of the first two forms, we use \(\Gamma(v), \Basic(v)\) and \(\Universals(v)\)
to denote the corresponding component of the sequent.\end{definition}

Informally,
\(\Gamma\) represents the set of assertions whose combined satisfiability we are checking.
\(\Basic\) and \(\Universals\) are sets of consistent assertions that we are using to build our model.
Each free element variable in these assertions corresponds (roughly) to a distinct element in the
model we are building (if one exists).

\begin{definition}[Tableaux]\label{def:tableau}
Fix a definition list \(\deflist\) for \(\psi\).
A \emph{tableau} for \(\psi\) is a possibly infinite labeled tree \((T,L)\).
We denote its nodes as \(\Nodes(T)\) and the root node is \(\rt(T)\).
The labeling function \(L \colon \Nodes \to \mathsf{Sequents}\)
associates every node of \(T\) with a sequent, such that the following conditions
are satisfied:

\begin{enumerate}
\def\labelenumi{\arabic{enumi}.}
\tightlist
\item
  \(L(\rt(T)) = \{ \sequent{\matches{x}{\psi}} \}\) where \(x\) is a fresh element variable;
\item
  For every \(s \in \Nodes(T)\), if one of the tableau rules in \(\SSS\) in Figure \ref{fig:guarded-tableau} can be applied (with respect to the
  definition list \(\deflist^\psi\)), and the resulting sequents are
  \(\seq_1,\dots,\seq_k\), then
  \(s\) has exactly \(k\) child nodes \(s_1,\dots,s_k\), and
  \(L(s_1) = \seq_1\), \dots, \(L(s_k) = \seq_k\).
\end{enumerate}

\end{definition}

\begin{proposition}For any sequent built using these rules, we cannot have both \(\matches{x}{\phi}\) and \(\matches{x}{\fnot\phi}\) in \(\Basic\)\end{proposition}

\begin{proof}The root node starts with \(\Basic\) empty, and therefore this invariant is
maintained. Basic assertions are added to \(\Basic\) only through the
\prule{resolve} rule which maintains the invariant.\end{proof}

\begin{proposition}There is a tableau for any guarded pattern\end{proposition}

\begin{proof}For any assertion that is not basic, there is some rule that applies.
For a basic assertion if either the assertion itself or its negation in in \(\Basic\),
then the \prule{conflict} or \prule{ok} rule applies.
Otherwise, there was some application of the \prule{exists}, \prule{app} rule
after which all variables in the assertion are in \(\Elements\).
We may build a tableau where the \prule{resolve} rule
is applied for this basic assertion after this \prule{exists}/\prule{app} rule.\end{proof}

\hypertarget{parity-game}{%
\subsection{Parity Game}\label{parity-game}}

As before, using this tableau we now build a parity game.
In this game, player \(0\) may be thought of as trying to prove the satisfiablity of the pattern,
and player \(1\) as trying to prove it unsatisfiable.

Each position in the game is a pair \((a, v)\) where \(a\) is an assertion
and \(v\) is either a sequent or \(\sat\).
If \(v = \unsat\), then \(a\) is of the form \(\matches{x}{\bot}\).
If \(\Gamma = \emptyset\), then \(a\) is of the form \(\matches{x}{\top}\).
Otherwise, \(a \in \Gamma\).

There is an edge from a position \((a_0, v_0)\) to \((a_1, v_2)\) if:

\begin{enumerate}
\def\labelenumi{\alph{enumi})}
\item
  \(v_1 = \unsat\) is a child constructed through (conflict)
  and (conflict-el), \(a_0 = a_1\). (same as above)
\item
  \begin{enumerate}
  \def\labelenumii{\arabic{enumii}.}
  \tightlist
  \item
    \(v_1\) is constructed from \(v_0\)
    using the (and), (or), (def), (mu), (nu), (\dapp) or (forall) rules
    and \(a_0\) was modified by this rule,
    and \(a_1\) is one of the newly created assertions.
  \item
    \(v_0\)'s child is created using (ok) or (ok-el)
    and \(a_0 = a_1\) and \(v_1 = \sat\).
  \item
    \(v_1\) is constructed from \(v_0\)
    using (choose-ex)
    and \(a_0 = a_1 = \alpha\) is the matched existential.
  \item
    \(v_1\) is constructed from \(v_0\)
    using (app) or (exists)
    and \(a_0 = \alpha\) and \(a_1\) is the an instantiation from \(\inst\).
  \end{enumerate}
\item
  \(v_1\) is a child constructed through any rule besides (conflict)
  and (conflict-el),
  \(a_0 = a_1\) is in \(\Gamma(v_0) \union \Universals(v_0)\)
  and \(\Gamma(v_1) \union \Universals(v_1)\).
  and \(v_1 = v_0\)
\end{enumerate}

\newcommand{\green}[1]{{\color{green}#1}}
\newcommand{\blue}[1]{{\color{blue}#1}}

A position \(p = (a, v)\) is in \(\Pos_0\) by rules with a {\color{green}green} rule. That is, if:

\begin{enumerate}
\def\labelenumi{\arabic{enumi}.}
\tightlist
\item
  \(v\)'s children were built using (or), (app), or (exists) rules
  and \(a\) was the assertion matched on by that rule; or
\item
  \(v\)'s children were built using (resolve); or
\item
  \(v = \unsat\)
\end{enumerate}

A position \(p = (a, v)\) is in \(\Pos_1\) by rules with a {\color{blue}blue} rule. That is if:

\begin{enumerate}
\def\labelenumi{\arabic{enumi}.}
\tightlist
\item
  \(v\)'s children were built using (and), (\dapp), (forall), or (choose-ex) rules
  and \(a\) was the assertion matched on by that rule;
\item
  \(v = \sat\)
\end{enumerate}

All other positions do not offer a choice, and so are arbitrarily assigned to \(\Pos_1\).

The parity condition \(\Omega\) is defined as follows:

\begin{itemize}
\tightlist
\item
  \(\Omega((e \in D_X, v)) = 2 \times i\) if \(D_i\) is a \(\mu\)-marker that is \(i\)th in the dependency order.
\item
  \(\Omega((e \in D_X, v)) = 2 \times i + 1\) if \(D_i\) is a \(\nu\)-marker that is \(i\)th in the dependency order.
\item
  \(\Omega((e \in \exists \bar x\ldotp \phi, v)) = 2 \times N + 1\) where \(N\) is the number of fixedpoint markers in \(\deflist\).
\item
  \(\Omega(a, v) = 2 \times N + 2\) otherwise.
\end{itemize}

Similar to Theorems\textasciitilde{}\ref{thm:qf-decidable}
we may prove that pre-models correspond to models and that guarded patterns are decidable.

\begin{theorem}If a guarded pattern has a tableau with a pre-model, then it is satisfiable.\end{theorem}

\begin{theorem}\label{theorem:validity}
If a guarded pattern has a tableau with a refutation, then it is unsatisfiable
and its negation is valid.\end{theorem}

\hypertarget{working-modulo-theories}{%
\subsection{Working modulo theories}\label{working-modulo-theories}}

The tableau presented in Figure \ref{fig:guarded-tableau} may be easily extended to handle
satisfiability modulo theorems for finite theories with guarded axioms.

First, we extend assertions to allow quantifying over its variable---i.e.
we allow assertions of the form \(\forall x \ldotp \matches{x}{\phi}\).
Next, we add a new tableau rule:
\[
\text{\prule{axiom}}\qquad
\pruleun{\sequent{\forall x \ldotp \matches{x}{\phi},\Gamma;\Basic;\Universals}}
        {\sequent{\inst \union \Gamma;\Basic;\Universals}}
\]

for each axiom \(\tau\) in the theory and \(x \in \free{\Gamma\union\Basic\union\Universals}\).

\hypertarget{complexity}{%
\subsection{Complexity}\label{complexity}}

\input{05-conclusion.tex}

\bibliography{./refs.bib}
\bibliographystyle{plainnat}
\end{document}
